\documentclass[12pt]{letter}

% \usepackage[britdate]{canterbury-letter}
\usepackage[britdate]{SU-letter}
\usepackage{times}
\usepackage{letterbib}
\usepackage{geometry}
\usepackage[round]{natbib}
\usepackage{graphicx}
\geometry{a4paper}
\usepackage[T1]{fontenc}
\usepackage[utf8]{inputenc}
\usepackage{authblk}
\usepackage[running]{lineno}
\usepackage{amsmath,amsfonts,amssymb}
% \usepackage[margin=10pt,font=small,labelfont=bf]{caption}

%\usepackage{natbib}
% \bibpunct[; ]{(}{)}{;}{a}{,}{;}

\newenvironment{refquote}{\bigskip \begin{it}}{\end{it}\smallskip}

\newenvironment{figure}{}

\position{Postdoctoral fellow}
\department{Department of Ecology, Environment, and Plant Sciences (DEEP)}
\location{Stockholm University}
\cityzip{Stockholm, Sweden}
\telephone{}
\fax{}
\email{alyssa.cirtwill@liu.se}
\url{http://cirtwill.github.io}
\name{Dr. Alyssa R. Cirtwill}

\newcommand{\myjournal}{\emph{New Phytologist}}

\begin{document}

\begin{letter}{\bf Professor Alistair M. Hetherington\\
               Editor-in-Chief, New Phytologist\\
               School of Biological Sciences\\
               University of Bristol\\
               Bristol BS8 1TQ, UK
                               }

% Borrowing heavily from the proposal

\opening{Dear Prof. Hetherington:}


	My co-authors and I are pleased to submit our manuscript for consideration as a research article in \textbf{New Phytologist}. 

	\begin{enumerate}

	\item What hypotheses or questions does this work address?
		\smallskip


		We test whether	more closely-related plants tend to share more pollinators and/or herbivores using a set of published plant-insect networks. We test this hypothesis both across whole communities and within plant families co-occurring in a community. We link the two scales of inquiry by investigating whether the strength of the whole-community relationship is related to the families present in the community.

		\smallskip


		\item How does this work advance our current understanding of plant science?
		\smallskip 


		By combining community-scale and family-scale analyses, we gain a richer and more detailed picture of the ways in which plant-animal interactions are structured with respect to evolutionary history. In particular, while we found that more closely-related plants do generally tend to share more interaction partners, we demonstrate the variety of patterns of conservation across families. This variability may help to explain the mixed results from earlier studies, especially those considering only a single network. Moreover, our approach takes into account both the numbers of shared interaction partners and the number of interaction partners which are not shared. We thus avoid discarding information, as happens when numbers of shared and not-shared interaction partners are converted to a single proportion. This more-detailed approach gives us more power to detect subtle trends.

		\smallskip

	
		\item Why is this work important and timely?
		\smallskip


		Plants' interactions with animals can significantly affect their fitness, both positively and negatively. These interactions might be conserved through stabilising selection or if key traits affecting the interactions are highly heritable, or they might not be conserved if there is selection to avoid overlap with close relatives or if distant relatives converge. The question of whether, and how strongly, plants' interaction partners are phylogenetically conserved has excited continuing interest, as it is likely an important step towards understanding the eco-evolutionary dynamics of plant-insect communities. Within this context, our work takes one step beyond considering whole-network trends to examine variation between plant families. We believe that this finer-grained approach will be critical in future efforts to find the mechanisms driving conservation of interaction partners.



		% \indent We note that our manuscript shares some methodological similarities with another important and timely article: Fontaine \& Th\'{e}bault, 2015 (Population Ecology). Due to the long gestation of this manuscript, we believe that the two manuscripts were developed independently. Nevertheless, we respect their prior publication and have taken care to refer to it in our manuscript. Our manuscript does, however, include a few methodological innovations which greatly add to its novelty. First, we define overlap in interaction partners using numbers of shared and not-shared partners (analogous to numbers of successes and failures in Bernoulli trials). This approach, as opposed to using only the number or propotion of shared interaction partners, effectively weights our data according to how much information is available and appears to be a more powerful tool to detect weak trends. Second, in addition to overall trends at the network level, we also test whether these trends are related to community composition and whether different plant families show different trends. These more-detailed analyses complement the usual network-level approach to give more insight into potential drivers of conservation of interaction partners. We believe that these additions to our methodology are enough to distinguish our work from that of Fontaine \& Th\'{e}bault. 



	\end{enumerate}

	Note that we submitted an earlier version of this manuscript to New Phytologist (NPH-MS-2016-21211), for which the revision period expired in August 2017. As the lead author was completing her PhD and transitioning to a postdoctoral appointment in a different country, we hope that the Editorial team will understand our failure to meet the original deadline. While we have taken the comments provided on the earlier version to heart, we now use a very different methodology and have revised the entire manuscript. We are therefore happy to have the current version stand on its own merits as a new initial submission, but are also prepared to provide a response to the previous round of comments if requested. Despite the delay, we hope that you will agree that the question we investigate remains of interest to the readership of \textbf{New Phytologist}.



    % My co-authors and I are pleased to resubmit a second revision of our manuscript  MEE-18-07-452:  \emph{A quantitative framework for investigating the reliability of empirical network construction} for consideration as a research article in \textbf{Methods in Ecology and Evolution}. Dr. Aaron Ellison, Senior Editor, noted that the reviewer who provided additional feedback was very positive about our revision but requested more discussion about prior choice. We have expanded our discussion to emphasise the need for careful prior selection and now refer to a demonstration of problems that can result from a poorly-chosen prior. We believe that these additions, although slightly exceeding the normal length restrictions of \textbf{Methods in Ecology and Evolution}, have improved the manuscript enough to justify the extra length. We thank the Editor for his encouragement and the Reviewer for his further thoughtful feedback.


    % Note that we uploaded an earlier version of this manuscript to biorXiv \\(doi: \emph{https://doi.org/10.1101/332536}). We have uploaded a copy of this version as CI\_preprint.pdf. The present version has been substantially re-organised and contains additional material not included in the pre-print.


    % We appreciate the opportunity to resubmit a further revision of our manuscript and thank you for your continued consideration. We remain confident that our manuscript will be of broad interest to the readers of \textbf{Methods in Ecology and Evolution}. We hope that you agree and eagerly await your reply.


\closing{Best regards,}

	% Reviewers:


\end{letter}
\end{document}


