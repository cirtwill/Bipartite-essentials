\documentclass[12pt]{letter}

% \usepackage[britdate]{canterbury-letter}
\usepackage[britdate]{SU-letter}
\usepackage{times}
\usepackage{letterbib}
\usepackage{geometry}
\usepackage[round]{natbib}
\usepackage{graphicx}
\geometry{a4paper}
\usepackage[T1]{fontenc}
\usepackage[utf8]{inputenc}
\usepackage{authblk}
\usepackage[running]{lineno}
\usepackage{amsmath,amsfonts,amssymb}
% \usepackage[margin=10pt,font=small,labelfont=bf]{caption}

%\usepackage{natbib}
% \bibpunct[; ]{(}{)}{;}{a}{,}{;}

\newenvironment{refquote}{\bigskip \begin{it}}{\end{it}\smallskip}

\newenvironment{figure}{}

\position{Postdoctoral fellow}
\department{Department of Ecology, Environment, and Plant Sciences (DEEP)}
\location{Stockholm University}
\cityzip{Stockholm, Sweden}
\telephone{}
\fax{}
\email{alyssa.cirtwill@liu.se}
\url{http://cirtwill.github.io}
\name{Dr. Alyssa R. Cirtwill}

\newcommand{\myjournal}{\emph{New Phytologist}}

\begin{document}

\begin{letter}{\bf Professor Alistair M. Hetherington\\
               Editor-in-Chief, New Phytologist\\
               School of Biological Sciences\\
               University of Bristol\\
               Bristol BS8 1TQ, UK
                               }

% Borrowing heavily from the proposal

\opening{Dear Prof. Hetherington:}


	My co-authors and I are pleased to submit our manuscript for consideration as a research article in \textbf{New Phytologist}. 

	\begin{enumerate}

	\item What hypotheses or questions does this work address?
		\smallskip


		We test whether	more closely-related plants tend to share more pollinators and/or herbivores using a set of published plant-insect networks. We test this hypothesis both across whole communities and within plant families co-occurring in a community. We link the two scales of inquiry by investigating whether the strength of the whole-community relationship is related to the families present in the community.

		\smallskip


		\item How does this work advance our current understanding of plant science?
		\smallskip 


		By combining community-scale and family-scale analyses, we gain a richer and more detailed picture of the ways in which plant-animal interactions are structured with respect to evolutionary history. In particular, while we found that more closely-related plants do generally tend to share more interaction partners, we demonstrate the variety of patterns of conservation across families. This variability may help to explain the mixed results from earlier studies, especially those considering only a single network.

		\smallskip

	
		\item Why is this work important and timely?
		\smallskip


		[[Not sure this is the best angle.]]
		Plants' interactions with animals can significantly affect their fitness, both positively and negatively. These interactions might be conserved through stabilising selection or if key traits affecting the interactions are highly heritable, or they might not be if there is selection to avoid overlap with close relatives or if distant relatives converge. By comparing plant families displaying different patterns of conservation of these interactions, we can predict which families may experience different types of selection pressures. Thus, we can use ecological data to guide future evolutionary research.


	\end{enumerate}

	Note that we submitted an earlier version of this manuscript to New Phytologist, for which the revision period expired in August 2017. We now use a different methodology from that presented in the initial submission, and we are happy to have the current version stand on its own merits as a new initial submission. We hope that you will agree that the question we investigate remains of interest to the readership of \textbf{New Phytologist}.



    % My co-authors and I are pleased to resubmit a second revision of our manuscript  MEE-18-07-452:  \emph{A quantitative framework for investigating the reliability of empirical network construction} for consideration as a research article in \textbf{Methods in Ecology and Evolution}. Dr. Aaron Ellison, Senior Editor, noted that the reviewer who provided additional feedback was very positive about our revision but requested more discussion about prior choice. We have expanded our discussion to emphasise the need for careful prior selection and now refer to a demonstration of problems that can result from a poorly-chosen prior. We believe that these additions, although slightly exceeding the normal length restrictions of \textbf{Methods in Ecology and Evolution}, have improved the manuscript enough to justify the extra length. We thank the Editor for his encouragement and the Reviewer for his further thoughtful feedback.


    % Note that we uploaded an earlier version of this manuscript to biorXiv \\(doi: \emph{https://doi.org/10.1101/332536}). We have uploaded a copy of this version as CI\_preprint.pdf. The present version has been substantially re-organised and contains additional material not included in the pre-print.


    % We appreciate the opportunity to resubmit a further revision of our manuscript and thank you for your continued consideration. We remain confident that our manuscript will be of broad interest to the readers of \textbf{Methods in Ecology and Evolution}. We hope that you agree and eagerly await your reply.


\closing{Best regards,}


\end{letter}
\end{document}


