\documentclass[12pt]{letter}

%\usepackage[britdate]{stockholm-letter}
 \usepackage[britdate,alyssa-signature]{stockholm-letter}
\usepackage{times}
\usepackage{letterbib}
\usepackage{geometry}
\usepackage[numbers,sort&compress]{natbib}
\usepackage{graphicx}
\geometry{a4paper}
\usepackage[T1]{fontenc}
\usepackage[utf8]{inputenc}
\usepackage{authblk}
\usepackage[running]{lineno}
\usepackage{amsmath,amsfonts,amssymb}
% \usepackage[margin=10pt,font=small,labelfont=bf]{caption}

%\usepackage{natbib}
% \bibpunct[; ]{(}{)}{;}{a}{,}{;}

\newenvironment{refquote}{\bigskip \begin{it}}{\end{it}\smallskip}

\newenvironment{figure}{}


\position{Postdoctoral fellow}
\department{Department of Ecology, Evolution, and Plant Sciences}
\location{Private Bag 4800}
\telephone{+64 3 364 2729}
\fax{+64 3 364 2590}
\email{alyssa.cirtwill@gmail.com}
\url{http://cirtwill.github.io}
\name{Alyssa R. Cirtwill}




\newcommand{\mytitle}{\emph{At a global scale, conservation of pollinators and herbivores between related plants varies widely across communities and between plant families.}}
\newcommand{\myjournal}{\emph{New Phytologist}}

\begin{document}

\begin{letter}{\bf Dr. Jana Vamosi\\
               Editor\\
               Proceedings of the Royal Society B\\
               6-9 Carlton House Terrace\\
               London, UK\\
               SW1Y 5AG\\
                }

\opening{Dear Dr. Vamosi:}


\closing{Regards,}


\end{letter}

\newpage

\setcounter{page}{1}

NPH-MS-2019-30803 At a global scale, conservation of pollinators and herbivores between related plants varies widely across communities and between plant families.
by Cirtwill, Alyssa; Dalla Riva, Giulio; Baker, Nick; Ohlsson, Mikael; Norström, Isabelle; Wohlfarth, Inger-Marie; Thia, Josh; Stouffer, Daniel

Dear Dr Cirtwill

I write you in regard to your submitted manuscript, NPH-MS-2019-30803 "At a global scale, conservation of pollinators and herbivores between related plants varies widely across communities and between plant families.”, which has now been reviewed. We found four reviewers (some original reviewers and some new) and they all generally enjoyed your revised manuscript. However, they also provided suggestions for some expanding certain sections and providing additional clarification. In particular, Reviewer #2 and #4 provide some detailed suggestions, especially with regard to keeping the aims and logic consistent, that will help strengthen the manuscript.

Although resolving these issues will require some restructuring, I think you should be able to handle these suggestions through revision (see below) and, ultimately, this manuscript will make a nice contribution to the literature.

Please carefully revise your manuscript and resubmit your revised version within 4 weeks. Please also include a summary of how you have incorporated the suggestions of the reviewers.

Thank you for submitting your work to New Phytologist.

Sincerely,

Jana Vamosi
Editor, New Phytologist

*****************************************************************
To revise your manuscript please click on the link below
*** PLEASE NOTE: This is a two-step process. After clicking on the link, you will be directed to a webpage to confirm. ***

https://mc.manuscriptcentral.com/newphytologist?URL_MASK=ebf0a265cb2d423bb35580e635386130

Alternatively log on to New Phytologist ScholarOne Manuscripts:
http://mc.manuscriptcentral.com/newphytologist

Enter your Author Centre.

Under the ‘My Manuscripts' list click on 'Manuscripts with Decisions'. From the list of manuscripts click the ‘create a revision' link, next to the manuscript you wish to submit a revision for. From here follow the instructions inputting you responses to the decision letter through to the uploading of your new files.

Please note you will be unable to make your revisions directly on the submitted ScholarOne Manuscripts version of your manuscript. Instead, you should revise your original word processor file, and then submit the revised version. When revising your manuscript, please prepare two versions: 1) the final revised manuscript and 2) the revised manuscript with revision marks showing. The latter file should be designated as a 'document with revision changes highlighted' when uploading.

In any correspondence regarding this manuscript, please include the manuscript reference number and copy the correspondence to New Phytologist Central Office (np-managinged@lancaster.ac.uk).
*************************************************************


Decision: Accept subject to revision

\clearpage

{\Large \bf Reply to Referee 1} [[done]]

	\begin{refquote}
	Having read the manuscript and the response to previous reviewers’ comments, I do not see any further changes that need to be made. Any concerns I would have had, have been addressed by the latest round of revisions. I did not even spot any typos!
	\end{refquote}

	\textbf{R:} We thank the referee for taking the time to read our manuscript and for the compliments to our revision and proof-reading.


\clearpage

{\Large \bf Reply to Referee 2}


	\begin{refquote}
	General comments

	The manuscript proposes to evaluate how phylogeny modulates species interaction patterns in plant-pollinator and a subset of plant-herbivory networks. In this regard, it explores an extremely relevant question for ecologists. However, it still needs work on the logic of the introduction section and methodological decisions need to be revised in order to achieve more confident results. The Introduction lacks focus and presents unclear aims, which leads to an unfocused discussion of the results, being one of the main weaknesses of the manuscript. Also, there is a lack of logical consistency between the abstract (methodologically-oriented) and the introduction and discussion sections (both biologically-oriented). On the other hand, we do not understand why authors do not incorporate ideas of species evolution and coevolution in a network context if you are trying to understand how phylogenetic constraints may modulate species interaction patterns in networks. Considering such exciting ideas may help to understand some of your results and may advance our understanding of  the importance evolutionary processes in determining species interaction patterns in communities.
	Another main point that you should revise across your manuscript is that you confound phylogenetic patterns with taxonomic patterns, for instance when thinking about congeneric species as closely related species (which is not necessarily true in several cases).

	\smallskip

	In what follows we share with authors our specific comments expecting that they will help to re-think the manuscript and get a stronger contribution.

	\end{refquote}


	The Referee(s?) seems to view our basic question and analyses positively, but is rather critical of our  writing both in terms of logical clarity and style. It is also clear that the Referee is more interested in coevolution than in any purely ecological network patterns. We thank the referee for taking the time to provide such extensive feedback and have done our best to incorporate as many of their suggestions as possible, although our ecological focus means that we have not made coevolution a major theme. Nevertheless, we now mention the potential for coevolution to affect network structure in the discussion. We address each comment in detail below (preceded by \textbf{R:}). It appears that the Referee wrote their specific comments before reading the manuscript in its entirety. Thus, some of the Referee's objections are in fact addressed at later points. In these cases, we have juxtaposed the Referee's comments to demonstrate this.


	1. General criticism of the language of the text

		\begin{refquote}

		Finally, the text needs English revision by the authors as we feel that some ideas are not clearly stated because of the way they are presented to readers. 

		\end{refquote}


		\textbf{R:} We would like to point out that both the lead and corresponding authors of this manuscript are native English speakers and that Referees 1 and 3 praised the quality of our writing. Of course technical correctness does not necessarily imply clarity of expression, and so we have revised the manuscript according to the Referee's suggestions. [[Given the Referee's own moderate to poor command of English, this is a hell of a thing to say!!]] If our loquacious verbiage has obfuscated any modicum of information, we apologise.


	2. Title lacks clarity

		\begin{refquote}
		Specific comments
		Title
		Your title still lacks clarity and does not reflect the main message you state at the beginning of your discussion (in the first sentence of it you state that you found broad support for the hypothesis that more closely related plants tend to have more niche overlap). We feel that this reflects lack of logical consistency across your manuscript, so re-thinking it should help to define your title.
		\end{refquote}


	3. Summary too methodologically-oriented in general [[done]]

		\begin{refquote}
			Summary
			General comments
			Your summary is too methodologically-oriented. In fact, when we were invited to be reviewers of your manuscript and received your abstract we thought that your paper represented a methodological contribution and it is not. After reading the comments that former reviewers sent you, your abstract seems to correspond to an older version of your manuscript. The thing is that by presenting an abstract that is methodologically-oriented when in fact you are not innovating in this sense, and having an introduction and a discussion that lack such methodological issue and are clearly more biologically-oriented can be read as that your work lacks focus.
		\end{refquote}

		\begin{refquote}
			Your 2nd paragraph is too technical: which is the biological meaning of it? Why are you going to compare pollination and herbivory networks? Why your datasets are so unbalanced? Why are you considering a wide range of pollination networks and a narrow one for herbivory networks? Why you consider shared and unshared partners? Your first point in the abstract doesn't introduce a methodological problem to be solve. (...) "and whether this relationship varies with the composition of the plant community"? What does it mean?
		\end{refquote}


		\textbf{R:} We regret that the Referee finds our summary too methodologically-orientated, but note that Referee 4 has requested that we include further methodological details in the summary. Moreover, in a later comment this Referee also requests more methodological details. To balance these conflicting comments, we no longer emphasise the methodological novelty of our study but have added the greater detail requested. As the other three Referees for this revision did not find the summary overly methodological in its original form, we are confident that the revised summary does not indicate that we lack focus. We also note that the extremely tight word limit for the summary does not allow us to explain why we have unbalanced numbers of pollination and herbivory networks. This information is contained in the methods. 


	4. Persistent mis-understanding of purpose and scope of within-family analyses [[done]]

		\begin{refquote}
		Your abstract also evidences your confusion with phylogenetic issues and taxonomic ones. For instance, your hypothesis is that more closely related plants will share more pollinators and herbivores than less closely related. Then, why searching for different relationships within families? This decision is completely arbitrary is one of the weakest point of the manuscript in general.
		\end{refquote}


		\textbf{R:} This comment indicates that the Referee has fundamentally misunderstood our manuscript based on the summary. Point 2 of the summary lists three separate analyses: testing for relationships between phylogenetic distance and partner overlap in \emph{all pairs of plants in a network, regardless of family}, testing whether the slopes of these whole-network relationships are related to the composition of the plant community in each network, and, as a supplement to these analyses, testing whether the slopes of these relationships differ if we consider only pairs of plants within the same family. We now number these three different analyses to avoid any further confusion. 


		As we are \emph{not} searching for differences between communities at the family level, but rather at all levels within the angiosperms, a number of the Referee's comments do not apply. For example, the comments below:


		\begin{refquote}
		3rd paragraph:  which is your biological argument to consider families as the taxonomic level to search for differences across communities?
		\end{refquote}


		\begin{refquote}
		Aims
		Why considering the phylogenetic signal at the plant families level? There is no introduction for this aim. Why focusing at such taxonomic level?
		\end{refquote}


		\begin{refquote}
			Line 318. Within-family conservation of niche overlap. Your result is almost expected so we think is important to justify biologically why you decided to look at the family level.
		\end{refquote}


	5. Very confusing comment complaining about paragraph 4 [[done]]

		\begin{refquote}
		4st paragraph "Considering factors affecting the dominant plant families within
		a community may be the key to understanding the distribution of interactions." So, relationships are more functional? Why focusing on plant families? we are not sure the logic of your thoughts can be followed (can the reader reach this conclusion? is it ok? we mean, if there are other eco-evolutionary processes determining species interactions why focusing in the dominant plant families would be informative? I feel the way this last paragraph is written isn't clear enough and is your main paragraph.
		\end{refquote}


		\textbf{R:} The Referee appears to have some deep-seated objection to any reference to plant families, but has not made the reasons for this objection at all clear. As noted above, we do in fact consider relationships between phylogenetic distance and interaction partner overlap at \emph{all} taxonomic levels, not just within families. Apart from this, we are confused by the remainder of the Referee's complaint. By 'dominant', we mean 'most species-rich in the focal community' rather than any kind of functional interpretation. Note that we do not address functions in the rest of the manuscript, so it is not clear why the Referee makes this leap. In order to avoid future confusion, we now replace 'dominant families' with 'species-rich groups'. We hope that this makes it clear that we are advocating researcher take a finer-scale perspective on phylogenetic signal in ecological networks.

		\begin{quotation}
		    \item The variety of relationships between phylogenetic distance and partner overlap in different plant families likely reflects a comparable variety of ecological and evolutionary processes. Considering factors affecting particular species-rich groups within a community may be the key to understanding the distribution of interactions.
		\end{quotation}


	6. Complaint about keywords absent from the abstract [[done]]

		\begin{refquote}
		Keywords
		Syndrome:  Why you do not mention such theory in your abstract?
		\end{refquote}


		\textbf{R:} The abstract/summary is under strict word limits, and we did not feel that inserting each keyword without a clear purpose could justify the elimination of other material. Given the Referee's confusion regarding our analyses, we are more convinced than ever that our decision to omit some keywords (which do appear in the introduction) in favour of spending more words explaining our methodology was correct. As we do not directly test pollination or defensive syndromes, we have removed these keywords from the list.


	7. Mistake about the focus of the introduction [[done]]

		\begin{refquote}
		Introduction
		General. You focus on the compromise between pollination and herbivory interactions but then you do not use this approach in your analyses.
		\end{refquote}


		\textbf{R:} Potential tradeoffs between encouraging interactions with pollinators and discouraging herbivores are just one of the possible reasons why networks may not display the expected relationship between relatedness and niche overlap. Note that we also propose several others in the same paragraph, including convergent evolution or selection to avoid pollen dilution. We therefore disagree with the Referee's perception that we focus on compromise between pollination and herbivory. It is beyond the scope of this manuscript to test all possible reasons why closely-related plants might not share interaction partners. This material is intended solely as background for readers who may not be familiar with arguments against the phylogenetic conservation of interaction partners.


	8. Deep disagreement with any link between syndromes and conservation of interaction partners [[done]]

		\begin{refquote}
		1st paragraph: We suggest this paragraph needs to be more functionally-oriented than phylogenetically-oriented. Syndromes may respond to the convergence of differently related organisms to a similar phenotype to attract the same pollinators. That is why plant syndromes can be more related with higher levels of taxonomic classification (orders) or to functionally defined groups like short and long-tongued bees, and not necessarily with sharing exactly the same pollinator species. Are you aware of articles showing that plant syndromes are phylogenetically conserved? We can think about many genera in which closely related plants show the same syndrome but bee or hummingbird syndromes are found within the family, so why should we expect that phylogeny has a strong signal? Or that more closely related plants will show more similar pollinators?
		\end{refquote}

		\begin{refquote}
			2nd paragraph: your reasoning is not clear for us. Syndromes imply the convergence of phenotypic traits into one common set of traits because of their functionality in increasing fitness (because of the selection pressures imposed by functional groups of pollinators that can or cannot be closely related). Even if traits are heritable, if many unrelated groups converge into one functional group of interaction partners, why you may expect more similarity among closely related species? On the other hand, can we expect modular structures based on your reasoning? Then, how you explain the high incidence of nested structures in both mutualistic and antagonistic networks? In fact, what has been proposed for coevolution in mutualistic networks is the convergence of species traits to those of the species forming the core (see Guimaraes et al 2017, Nuismer et al. 2018). In antagonistic networks your line of reasoning may apply, but this will clearly depend on the intimacy of your network as most herbivory networks tend to show nested structures too. Your reasoning is based on pairwise coevolution and does not incorporate ideas of evolution and coevolution in networks, which may completely alter your predictions about niche overlap.
		\end{refquote}

		\begin{refquote}
			Lines 65-65. You wrote: Unrelated plants might also converge upon similar phenotypes, attracting a particularly efficient or abundant pollinator (Ollerton, 1996; Wilson et al., 2007; Ollerton et al., 2009; Ibanez et al., 2016). This is exactly our point: pollination syndromes may reflect trait convergence among unrelated species.
		\end{refquote}


		\textbf{R:} As the Referee notes, we do address the possibility of convergence later in our introduction. We do not make any claim as to whether the putative pollination syndromes actually reflect plants with common sets of pollinators (the evidence we are aware of is rather mixed). Since the Referee appears to object violently to this very common idea in the literature, and because syndromes do not form a major part of our analysis, we have removed it as a keyword and rephrased our introduction. It now no longer references pollination syndromes. Hopefully this will avoid other readers becoming fixated on the validity or otherwise of the syndrome idea and enable them to take our manuscript on its own merits.


		Lines XX-XX:

		\begin{quotation}

			Plants with different defences 
		  	(e.g., thorns vs. chemical defences) may deter different groups of 
		  	herbivores~\citep{Ehrlich1964,Johnson2014}, and pollinators with similar traits are often expected to attract similar sets of pollinators~\citep{Waser1996,Fenster2004,Ollerton2009}.

		\end{quotation}


	9. Request to include interaction intimacy [[done]]

		\begin{refquote}
		Lines 35-39. Maybe you should use the idea of interaction intimacy, since it clearly will determine the level of selective pressure landscape perceived by a given species?
		\end{refquote}


		\textbf{R:} We have no information on interaction strengths or intimacies, and so cannot test whether this is affecting our results. However, we are happy to include the idea as yet another reason why plant communities may or may not show conservation of interactions. We have added a brief description of intimacy as follows:


		Lines XX-XX:

		\begin{quotation}

			In mutualistic networks, animals often show a stronger phylogenetic signal in their partners than do plants~\citep{Rezende2007a,Chamberlain2014,Rohr2014,Vamosi2014,Lind2015,Fontaine2015} (but see~\citet{Rafferty2013} for a counterexample). In antagonistic networks, however, actively-foraging consumers tend to show less phylogenetic signal than their prey~\citep{Ives2006,Cagnolo2011,Naisbit2011,Fontaine2015}. In part, this may be related to different degrees of interaction intimacy (dependence of one predator on another), which appears to contribute to network structure in mutualistic, but not antagonistic, networks~\citep{Guimaraes2007,Ponisio2017}. 

		\end{quotation}


	10. Criticisism of methodological changes required by previous reviewers [[done]]

		\begin{refquote}
		Methods
		Dataset
		Several networks can be reviewed in the literature or are available in different websites. Why authors decide not to consider several plant-herbivore networks but have no filter to separate plant-pollinator networks? The reason of why focusing in plant-pollinator and plant-herbivory networks is not clear for us as well why authors choose to select just one kind of herbivory interaction and as a consequence have a highly unbalanced design. Such decision seems to be crucial as it may modulate the results and conclusions they achieve. By eliminating the herbivory interactions that are more phylogenetically constrained their result that pollination interactions may be more conserved than herbivory interactions seems to be a by-product of such decision.
		\end{refquote}


		\textbf{R:} We restricted our selection of herbivory networks in response to a previous round of review. Without more details why (or if) the Referee believes that the herbivory networks we removed should be included, we have opted to continue following that earlier feedback. We would have appreciated specific tips for additional herbivory networks, but since the Referee does not provide any specifics there would appear to be little we can do to address this comment.


	11. Request to change "insects consuming leaves" to "chewing insects" [[done]]

		\begin{refquote}
			Lines 93 to 96. We think that sap-sucking, leaf mining, and galling insects are also insect consuming leaves. Maybe you could change "insect consuming leaves" by "chewing insects"
		\end{refquote}

		\textbf{R:} The term "insects consuming leaves" was recommended by an earlier reviewer, but we agree that leaf mining also constitutes leaf consumption (this is not necessarily the case for sap-sucking and galling, which is less leaf-specific). We have changed these lines as suggested.


	12. Recommendation to continue using the same methodology [[done]]

		\begin{refquote}
			Measuring niche overlap
			Why you used absolute numbers to measure shared and unshared pollinators and not the Jaccard index? Or, instead, why you did not use a binomial regression considering the absolute number of the whole set of pollinators that can be shared ? You should use generalized linear models with a binomial distribution considering the total number of pollinators and shared pollinators to give more weight to those species sharing more pollinators in a larger sample (generalists s for instance). It is not clear for us if your analysis considers the full number of species over which you obtain the percentage of shared species.
		\end{refquote}

		\textbf{R:} We did use a Jaccard index, as stated in the first and second sentences under the sub-heading \textbf{Calculating niche overlap between communities}. The Jaccard index is calculated using the number of shared and un-shared interaction partners for a pair of plants, as stated in equation 1. We also did use a generalised linear model with binomial error distribution, as stated in the \textbf{Statistical analysis} section. We chose this methodology for exactly the reasons the Referee suggests, to give more weight to generalists sharing many interaction partners. This is stated in lines XX-XX (quoted below). We are glad that the Referee agrees with our choice in methdology.

		\begin{quotation}

			We wished to give more 
		    weight to species sharing a large number of interaction partners as well as 
		    those sharing a large proportion (i.e., to emphasise pairs of generalists 
		    sharing most of their interaction partners over specialists sharing a single 
		    interaction partner). Note that species sharing a large \emph{number} of interaction partners may not share a large \emph{proportion} of interaction partners if the number of interaction partners that are not shared is also large.

		\end{quotation}


	13. Request to include plant species as a fixed effect [[done]]

		\begin{refquote}
			176-178. Why you do not consider using plant species as another fixed factor to break the lack of independence? This is because generalist species may have more weight in your analyses than other species. It would be interesting to see how your results differ from an analysis considering such approach.
		\end{refquote}

		\textbf{R:} It sounds as though the Referee would like us to include plant species as a predictor in our regressions. Due to the fact that species generally appear in few networks (many in only a single network), this would result in near singularity of our models. We would therefore be completely unable to observe any patterns as species identity would explain essentially all variation. If the Referee is referring to a potential effect of network size, we did in fact test this (see equation 4 and description). If the Referee did not mean either of these possibilities, we apologise but are unable to understand this comment.


	16. Un-justified request to include connectance [[just address here]]

		\begin{refquote}
			198-209. Why did you decide to control just for network size and do not consider network connectance? You provide methodological information to explain your decision but there is no biological argument explaining why network size may influence the relationship between phylogenetic relatedness and niche overlap.
		\end{refquote}

		\textbf{R:} We did not include connectance because we can offer no justification (biological or methodological) why the overall connectance of the network should affect the strength of the relationship between relatedness and niche overlap. It is possible that there might be an effect of connectance on the intercept of this relationship (as we would expect higher overlap in general when connectance is high), but we see no particular reason why connectance should change how closely-related species differ from distantly-related species in the same network. Since the Referee also does not provide any reasoning behind this request, we have opted to leave our analyses as-is.


		[[Maybe I can add a bit of a biological argument that network size can affect overlap? I'm not sure what methodological but not biological means here.]]


	17. Confusion about motivation for community composition analyses [[done]]

		\begin{refquote}
			Line 210 Linking network level trends and community composition

			(related with this part of the methods but referring to the Introduction section) Which are the hypothesis for this analysis? How do you think that family composition (and no species composition) could affect the relationship between phylogenetic  distances and niche overlap?
			In the analysis it is not clear for us which are the dependent and independent variables. Is the independent variable a matrix with coefficients of similarity between networks?  Do you want to explore if more similar networks have a more similar relationship (slope) between phylogenetic distance and niche overlap? Or you want to investigate if the slopes are more o less strong depending on the local composition of communities (for each network separately) ? If this is the case you could compare slopes with a local diversity index (e.g. Shannon) instead of a beta diversity index.
		\end{refquote}


		\textbf{R:} The Referee was correct with their first idea. We want to explore whether more similar networks have more similar slopes of partner overlap against phylogenetic distance. Once again, we are pleased that the Referee agrees with our analysis. We did this because different plant families likely experienced different levels of convergence, coevolution, have different levels of intimacy with their partners, etc. We now explicitly state this in the introduction.


		Lines XX-XX:

		\begin{quotation}

		  Here we investigate how overlap in interaction partners between 
		  pairs of plants (henceforth ``niche overlap'') varies over 
		  phylogenetic distance and how this differs between plant families. 
		  Whereas previous 
		  studies have focused on the presence or absence of phylogenetic
		  signal across entire networks, we take a pairwise perspective in
		  order to obtain a more detailed picture of how plant phylogeny
		  relates to network structure. As different plant families may have experienced different degrees of coevolution, convergence, etc., we also complement analyses with entire networks with comparisons among plants in the same family within a network. 
		  This novel perspective allows us to investigate the relationship between phylogenetic distance and partner overlap at different scales. 

  		\end{quotation}


	18. Critique of manuscript structure [[not sure  how to address]]

		\begin{refquote}
			Results
			General: Please, do not include discussion in your results, it may confound readers.
		\end{refquote}

		\textbf{R:} Where, specifically, does the Referee feel that we have included discussion in our results? This vague and somewhat condescending feedback is not helpful.


	19. Possible complaint about mixing numbers and percentages?

		\begin{refquote}
			Line 262. Your first sentence of results do not correspond with what follows. We will suggest you use most networks, or most herbivory networks and half pollination networks, or using the percentages you present in the discussion section.
		\end{refquote}

		\textbf{R:} This comment is in such poor English as to be nearly unintelligible. It is not clear whether the Referee is suggesting that we repeat our analyses using half of the pollination networks (without any clear reason as to why) or complaining that we have mixed qualitative descriptors, numbers, and percentages. We assume the latter as we are able to address this complaint. [[check what's going on]] 


	20. Confusion about methodology

		\begin{refquote}
			Line 310. How was community composition measured? Is difficult to evaluate the quality of this result because you haven't described this methodology.
		\end{refquote}

		\textbf{R:} Community composition is the set of plant species that make up a community. It was defined by the researchers constructing each network according to the specific methodology employed in each case. [[Add a parenthetical definition]]


	21. Confusion about permutation methodology

		\begin{refquote}
			Minor comments on Results
			297-298 You permuted interactions or phylogenetic distances?
		\end{refquote}


		\textbf{R:} [[double-check and provide a line ref]]


	22. Request to change direction of relationship?
		\begin{refquote}
			Fig 1. "...the probability of a pair of plants sharing an interaction partner increased with increasing phylogenetic distance..." please, write decreased instead of increased.
		\end{refquote}


		\textbf{R:} Sweet Satan, I hope this isn't a typo?


	23. Difference between minimum and maximum slopes not visible

		\begin{refquote}
			Fig 2. It is no clear for us. Minimum and maximum slopes obtained from 999 permutations are the same value?
		\end{refquote}


		\textbf{R:} [[Check that caption states that min and max are very close together since permutation completely destroyed pattern.]]


	24. Missing word in line 328

		\begin{refquote}
			Line  328: "...increasing phylogenetic distance in 12...", what is 12? the number of families?
		\end{refquote}

		\textbf{R:}


	25. Sundry complaints about table numbering and figure referencing

		\begin{refquote}
			Table number seem to be erroneous as you first mention table 2 and then table 1.

			\smallskip

			Figure 3 is only referenced in the text when talking about herbivory networks but not for pollination networks.
		\end{refquote}

		\textbf{R:}


	26. Blanket demand for more evolution content

		\begin{refquote}
			Discussion
			General
			We suggest you include ideas about evolution and coevolution in networks in your discussion. It is important to recognize that network structure is in part a consequence of the evolutionary process and that we cannot think in the pairwise-coevolutionary framework as the only one modulating trait evolution in species rich assemblages.
			You state that the fact that the proportion of specialist species may explain differences between plant-pollination and plant-herbivore network patterns and you discuss the mathematical consequences of such difference but you should discuss the ecological and evolutionary causes and consequences of having higher specialization in pollination networks. This is also related with your methodological decision of excluding more specialized herbivores, isn't it?
		\end{refquote}

		\textbf{R:} THIS IS NOT AN EVOLUTION PAPER! WE HAVE NO DATA TO INFER ANYTHING ABOUT THE EVOLUTION OF THESE SYSTEMS!


	27. Debate over the meaning of "broad"

		\begin{refquote}
			338-339. You state We found broad support for the hypothesis that more closely-related pairs of plants have a higher degree of niche overlap. This is a very strong sentence that is not supported by your results. You found evidence in 55% of plant-pollination networks (almost half of your dataset). In fact, it is interesting to note that even the slope is more pronounced in mutualistic networks you found higher support in herbivory networks, as a higher percentage of networks showed a tendency.
		\end{refquote}


		\textbf{R:}


	28. Another complaint about sub-setting the herbivory networks, possibly addressable in text

		\begin{refquote}
			351-352 The herbivory networks did not contain as many obligate specialists, but we note that many herbivorous insects are oligotrophs which consume only a few closely-related hosts (Novotny & Basset, 2005; Yguel et al., 2011). This sentence is confuse because you have a subsample of herbivory networks and then this seems to be a by-product of your study design associated with your decision of removing networks that may show strong phylogenetic signal.
		\end{refquote}

		\textbf{R:} 


	29. Claims that congeneric species are not closely related

		\begin{refquote}
			357-359 This suggests that conservation of interaction partners among closely related plants (e.g., congeners or members of the same subfamilies) is more important than phylogenetic signal from deeper within the phylogenetic tree. Congeneric species can be not closely related, please revise sentences like this one across the manuscript because you are confounding phylogeny with taxonomy.
		\end{refquote}


		\textbf{R:} Does the Referee really mean to suggest that plants in the same genus are less closely-related than plants in different families? Without some sources to substantiate this, we find this claim very far-fetched!


	30. Another demand to discuss intimacy [[possibly addressable, since we did add a tiny reference in intro]]

		\begin{refquote}
			361-365. How interaction intimacy may explain your results? Please, discuss your results in light of the interplay between interaction type (pollination, herbivory) and intimacy (see Hembry et al. 2018).
		\end{refquote}


		\textbf{R:}


\clearpage

{\Large \bf Reply to Referee 3} [[Still to do: title, add a bit about how this could be relevant for conservation.]]

	\begin{refquote}
		The study is well designed and the manuscript well written, and the authors clearly demonstrate their comprehensive grasp of the subject and the applied statistical analyses.
		My only more general comment is whether authors have considered potential applications of such phylogenetic approaches for ecological restoration? If closely-related species provide similar functions in ecological networks, they could be used interchangeably to restore ecological interactions/functions. This is an exciting potential application of phylogenetic data in biodiversity conservation and restoration, and could be touched upon/discussed in this manuscript.. 
		Otherwise, I have only minor comments.
	\end{refquote}

	\textbf{R:} We thank the Referee for their compliments to our approach and to our writing, as well as their more detailed comments (addressed one-by one below; replies preceded by \textbf{R:}). We had not considered this potential application for ecological restoration but are very pleased to have it pointed out. It is often difficult to see how theoretical work may relate to conservation practice, so we are glad to take the opportunity to make a link here. [[Add a bit of this, I think we have room]]

	1. Title is confusing

		\begin{refquote}
			Title is confusing – difficult to pick out key message. And given that your analyses broadly give support for effects of phylogenetic distance on niche overlap, why not focus on this message, rather than variability at among individual networks/plant families
		\end{refquote}


		\textbf{R:} 


	2. Request to state that networks are qualitative rather than quantitative [[done]]

		\textbf{R:} The Referee is correct that the difference between qualitative and quantitative networks is crucial for the correct ecological interpretation of said networks. We thank them for pointing out that the distinction wasn't clear in the previous version. We now specify that our networks are qualitative at the beginning of the methods and results and the end of the discussion. We hope that these few reminders will be enough to remind readers about the structure of our networks.

		Lines XX-XX (methods):

		\begin{quotation}

			All networks were qualitative, recording the presence or 
		    absence of interactions rather than their strength.

		\end{quotation}


		Lines XX-XX (results):
		
		\begin{quotation}

			Note that, as our networks 
		    are qualitative, these results refer only to the number of shared interaction
	    	partners rather than to the quantitative strength of competition.

    	\end{quotation}


    	Lines XX-XX (discussion):

    	\begin{quotation}

		  Although here we considered only the presence or absence of interactions,
		  (i.e., qualitative networks)

    	\end{quotation}


	3. L298 – Please be careful with use of “lesser/greater” “more negative or positive than expected from permuted networks” [[done]]

		\textbf{R:} We appreciate that this desceription could be unclear, and have revised this line as suggested. It now reads:

		Lines XX-XX:

		\begin{quotation}

			the observed 
		    slope of the relationship between phylogenetic distance and interaction 
		    partner overlap was always more extreme (i.e., always more negative or 
		    always more positive) than that obtained in the permuted networks (Fig. 2).

	    \end{quotation}


	4. Typo in line 322 [[done]]
		
		\begin{refquote}
			L322 – how was this result obtained if insect phylogenies were not caclculated? Do you mean “more closely-related plants in pollinator networks”?
		\end{refquote}


		\textbf{R:} The Referee is correct and we apologize for this extremely confusing typo. We have corrected this line as suggested.


\clearpage


{\Large \bf Reply to Referee 4}

	\begin{refquote}
		Understanding what drives the structure of networks of interactions is key to predict the structure of plant-insect communities (e.g., for conservation or restoration purposes, or to support benefit organisms in agriculture).
		The present study focuses on the contribution of the evolutionary history to the interaction structure, and makes a significant contribution to the understanding of the variability of the phylogenetic signal in plant-insect networks. Here, the strength of the phylogenetic signal is actually quantified while it usually relies on correlation scores, and the effect of plant composition and the identity of plant families are examined to assess whether they could explain the variability of trends described in the literature.
		The dense writing turns out to be a double-edge sword: a lot of information on drivers of phylogenetic signals is compiled in here, but the text is sometime hard to follow. Hence, my following comments mostly pertains to the form, and I trust the authors should be able to address them fairly easily.
	\end{refquote}


	\textbf{R:} We thank the Referee for their compliments to our analysis and for taking the time to review our manuscript. We appreciate that our writing can be dense at times and appreciate the Referee pointing out places where we may clarify our prose. Below, we respond to each comment in detail (preceded by \textbf{R:}). We are confident that revising the manuscript in response to the Referee's comments has improved the manuscript.


1. Confusion about description of two dependent variables in analyses

	\begin{refquote}
		* Modelling the relationship between niche overlap and phylogenetic distance
		I did not managed to fully grasp the statistical analysis right away (I reckon I'm not a statistical master, but this may be the case of other readers), and I think it was mostly due to the way the methods are described. The writing is nice, but some tiny details are missing while they could help the reader to understand more quickly how the authors distinguish the use of the {M_ij, U_ij}-tuple from that of J_ij.
		Based on l. 143-167, I understand that the number of shared interactions M_ij is modelled with a binomial distribution, with the number of trials being M_ij+U_ij, and the probability of success being w_ij (M_ij ~ B(M_ij + U_ij, w_ij)). The {M_ij, U_ij}-tuple would appear in R as follows:
		model ← glm(cbind(M_ij, M_ij+U_ij) ~ d_ij + ..., family = “binomial”)
		But, can we actually say that there are two dependent variables? To me, it is the number of shared interactions that is modelled as a binomial process, not both the number of shared and unshared interactions.

		\smallskip

		p. 5, l. 69-71 « In either case, dissimilarity (…). » This sentence helped me to understand why it is important to look at unshared interactions. To be better showcased ?

	\end{refquote}


	\textbf{R:} The number of shared and unshared interactions are both dependent variables included in the analysis. The number of shared variables is indeed the key variable, but the number of unshared interactions is also important as it indicates how extraordinary a particular number of shared interactions is. The binomial process incorportates both quantities; converting the numbers of shared and unshared interactions to a proportion and weighting the observations by the total number of interactions. [[Need to address this in the text]]


2. Request to state that we use the Jaccard index more explicitly [[done]]

	\begin{refquote}
		I suppose that the analysis described in paragraph starting from l. 168 corresponds to actually modelling the Jaccard index (J_ij ~ ...). If this is right, I think this should be specified in the second paragraph (no direct mention of the Jaccard index is currently made, except the proportion l. 170 between brackets although it is not exactly written as in eq. (1)), while in the first paragraph, an equation should specify that M_ij ~ B(M_ij + U_ij, w_ij)).
	\end{refquote}


	\textbf{R:} Yes, this analysis is also using a Jaccard index, but we see how the previous version did not make this clear. We now state that both the tuple and proportional glm's use Jaccard dissimilarity. We also re-iterate the logic behind the tuple-based formulation in this paragraph.


		Lines XX-XX:


		\begin{quotation}

			To demonstrate the power of defining $\omega_{ij}$ as a tuple of $M_{ij}$ and $U_{ij}$, we repeated the above analyses using a Jaccard index based only on the proportion of interaction partners that are shared (i.e., $\omega_{ij}$ = $M_{ij}$/[$M_{ij}+U_{ij}$]). Note that while the proportion of shared interaction partners is the same in both cases, the tuple formulation gives more weight to plants with many interaction partners as these provide more information. When comparing the two approaches 
		    we observed similar trends but, notably, the tuple definition of $\omega_{ij}$ had greater power to detect weak relationships (\emph{Supporting information 2}). We therefore show only the results when defining $\omega_{ij}$ as a tuple in the main text.

		\end{quotation}


3. Request to make non-independence of species within each community explicit. [[done]]

	\begin{refquote}
		Another point pertains to the definition of “trials” in the binomial-distributed process. If I get it right, these correspond to the union of species i and j interactions (of size P_i + P_j -2M_ij = M_ij + U_ij). However, these are not independent for reasons highlighted in the introduction (e.g., competition, facilitation between plants and/or insects). This may question the robustness of the analysis, but testing the effect of the plant community composition (negative) allows to verify this. This should be specified, I think.
	\end{refquote}


	\textbf{R:} We actually address the non-independence of trials (pairs of plants) by calculating $p$-values based on permutations of the observed networks (as opposed to assuming independent, binomially-distributed trials, which is certainly not the case). To emphasize our explanation of this, we have reordered the text and placed our discussion of the permutation methods under the sub-heading 'accounting for non-independence'. We hope that this will calm reader's worries about the validity of our conclusions.


4. Permuting the permuted networks is confusing

	\begin{refquote}
		* Evaluating type I and type II errors: Permutations of permutations
		Paragraph p. 10, l. 187-197 is a bit confusing because of phrasings such as “permutations/permuting of the permuted networks”. I don't have a better way to name this second round of permutations, but the overall is difficult to follow and could be shortened while still conveying the same message.
		Actually, I find the comment in Supplementary information 3 clearer, although no reference to Supplementary information 3 is made.
		Besides, as slopes of regressions in permuted networks are tightly grouped around zero (Fig. 2, thin lines indicating 
		extrema of slopes distribution overlapping each other), so type II errors do not seem to be a big problem here.
	\end{refquote}


	\textbf{R:} We agree with the Referee about the difficulty of describing this procedure. [[Can I just swap in whatever is in S3?]] Despite the difficulty in describing our process for finding type II error verbally, we are pleased that the Referee found the end result (Fig. 2) interpretable. [[Maybe I can move all of the type II error testing to SI?]]


5. Suggestion to consider phylogenetic diversity instead of number of plant pairs

	\begin{refquote}
		* Testing the effect of network size on the existence of a phylogenetic signal
		Here, it is specifically the number of plant pairs that is studied. I wonder whether the phylogenetic diversity of plants would not be more adapted here. Is it overlooked because similar among networks?
		I think knowing more about this effect would further illuminate why the strength of the phylogenetic signal varies between networks and would be complementary to the analysis testing the effect of plant community composition.
	\end{refquote}


	\textbf{R:} Umm... what does this mean? [[Probably going to go with "beyond the scope"]]


6. Request for consideration of multiple testing

	\begin{refquote}
		* Testing within-family existence of a phylogenetic signal
		Examining the strength of phylogenetic signal within plant families is an interesting idea. However, this leads to multiple testing which is prone to type I errors and I could not find mention of correction for this multiple testing in the text. There are actually high chances that there is at least one significant test in each table (1 – (1 – alpha)^N = 0.36 and 0.91 respectively, alpha = 0.05). Yet, the authors can indicate that there are low chances to have that many significant phylogenetic signals (using the probability mass function of the Binomial distribution, see Moran 2003 in Oikos).
	\end{refquote}


	\textbf{R:} [[Wasn't this the point of looking at type I errors? Or something?]]


7. Suggestion to mention sampling completeness as a potential explanation for specialists (discussion)

	\begin{refquote}
		On the effect of specialist pollinators

		I think that Reviewer #3 made an interesting remark on specialist pollinators, and I agree with the authors on the impossibility to assess sampling completeness on the network of this database. However, I think this should be mentioned in the text, at least to invite future works to provide their sampling effort when publishing new empirical data (hence, helping to improve future analyses).
	\end{refquote}


	\textbf{R:}


8. Discussion of potential tradeoffs between interactions could be sharpened (discussion)

	\begin{refquote}
		On plant families involved in different types of interactions

		p. 19-20, l. 432-448: This paragraph on systems for which both pollination and herbivory were sampled is a bit fuzzy. Talking about how diverse types of interactions can work together is a nice way to bring back the study to a broader context, though.
		First, I think the sentence of simultaneous selection pressure of pollination and herbivory could be a bit clarified (l. 439-442, “Plants may not be able to respond (…).”). Strauss & Whitthall’s chapter in the book “Ecology and evolution of flowers” gives a good insight on how these selection pressures can act together or antagonistically in shaping plants phenotypes (and eventually interactions, although not specifically discussed in their chapter).
		Second, the structure of interaction patterns could be discussed with more references than Sauve et al. (2016). Astegiano et al. (2017) make an interesting point on the asymmetry of interaction patterns in such systems (using Pearse and Altermatt (2011)’s data on Lepidotera diet, and citing the Norwood farm data set of Pocock et al. 2012, and the Donana network in Melian et al. 2009).
		Finally, the existence dynamical drivers of the structure of pollination and herbivory is mentioned. Sauve et al. (2014, 2016) indeed suggest that the way pollination and herbivory are distributed on the plant community may matter for community stability, and they studied one case where this structure is indeed not random. Yet, they do not say that these communities should be more stable (compared to which ones since plants undergo both herbivory and pollination in nature?). Whether the interconnection structure arises because it provides greater stability (as discussed by Sauve et al. 2016), or as a by-product of the assembly process (Maynard et al. 2018, Ecol. Lett.) remains to be discussed for networks combining different types of networks.
		Despite the length of this last comment, I don’t mean that the authors should develop proportionally this paragraph, but rather sharpen it.
	\end{refquote}


	\textbf{R:} - merging interaction types is a good way to bring back to broader context :) - line 439-442 could be clarified based on Strauss & Whitthall (if I can find it) - more references for structure of interaction patterns: e.g., Astegiano et al. - be more tentative for dynamical drivers (apparently Sauve et al. don't really discuss stability). Could be a by-product of assembly process (Maynard et al., 2018)


Minor comments

9. Request for additional detail in summary point 2 [[done]]

	\begin{refquote}
		In the abstract, item 2 « shared and unshared interactions »: if it does not breach the word limit, I think a few words specifying why this is important would better showcase the study, and help to understand how it contributes to the study novelty.
	\end{refquote}

	\textbf{R:} We appreciate the Referee's suggestion. Although we did not have much room to expand the summary, we have added a short parenthetical explanation of why considering both shared and unshared partners is helpful. This point now reads:

	\begin{quotation}

		\item We quantify overlap of interaction partners for all pairs of plants in 59 pollination and 11 herbivory networks based on the numbers of shared and unshared interaction partners (thereby capturing both proportional and absolute overlap). We test 1) for relationships between phylogenetic distance and partner overlap within each network, 2) whether these relationships varied with the composition of the plant community, and 3) whether well-represented plant families showed different relationships. 

	\end{quotation}


10. Incorrect referencing of Sauve et al., 2016 [[done]]

	\begin{refquote}
		p.3, l. 1-2, citation of Sauve et al. 2016 (Ecology) to say that interactions between plants and animals are critical to plants’ life cycle. I’m not sure this is the best pick as Sauve et al. rather look at how pollination and herbivory interactions are distributed in the plant community and how it may affect community stability. The following references are adequate though.
	\end{refquote}

	\textbf{R:} We have removed this reference as suggested.


11. Possible typo in page 3, lines 9-10 [[done]]
	
	\begin{refquote}
		p. 3, l. 9-10 « A plant’s traits are also (...) », I think there is a typo here. Shouldn’t « a » be removed ?
	\end{refquote}


	\textbf{R:} We do not believe that the original formulation was incorrect, but agree that it was inelegant. We have revised this line to read:
	\begin{quotation}
		Plant traits are also likely to determine \emph{which} specific pollinators and herbivores interact with a particular plant.
  	\end{quotation}
  	We hope that this revised version reads better.


12. ?Suggestion to remove long list of citations in lines 22-25? [[done]]

	\begin{refquote}
		p. 3, l. 22-25 : There is a lot of citations here. I agree they all point to « mixed results » but this pertains to different aspects of phylogenetic signal. Some highlight differences between mutualistic and antagonistic systems, other rather focus on difference of phylogenetic signal between trophic levels, or even focus on subset of interaction networks. I think, it would be more relevant to the reader to cite these different works to explain how mixed results are (as done in the text following this long citation list).
	\end{refquote}


	\textbf{R:} We had intended these citations to showcase a selection of researchers addressing different aspectsof phylogenetic signal, but appreciate how including them all together could be confusing. We have removed the large list so as not to distract from the more specific citations in subsequent lines.


11. Better showcase sentence on page 5, lines 78 [[done]]

	\begin{refquote}
		p. 5, l. 78 « a pairwise perspective » : This contributes to the manuscript novelty. Maybe this should be further specified?
	\end{refquote}

	\textbf{R:} We thank the Referee for noticing this line, and have added a bit more detail as to how this pairwise perspective allows us to consider both whole-network trends and trends within families.


	Lines XX-XX:

	\begin{quotation}

		Whereas previous 
	  studies have focused on the presence or absence of phylogenetic
	  signal across entire networks, we take a pairwise perspective in
	  order to obtain a more detailed picture of how plant phylogeny
	  relates to network structure. This novel perspective allows us to investigate the relationship between phylogenetic distance and partner overlap at different scales. 

	\end{quotation}


12. Suggestion to make a figure showing distribution of webs, or at least to add countries to the list in SI [[done]]

	\begin{refquote}
		p. 6, l. 89-90 « These networks span a range of biomes (…). » In supporting information, the countries are not indicated in the list. Alternatively, and to emphasize on the title following Reviewer #2 suggestion in the previous round of review, a figure showing the spatial distributions of the 59 + 11 network on a world map would be welcome here.
	\end{refquote}


	\textbf{R:} We have added the countries or regions (e.g., islands associated with mainland countries, such as Greenland and Denmark) to Table 1 in the SI. We hope that this will provide enough clarification about the spatial scale of our dataset. For readers who are interested, we now refer both to the original sources and from an online database (the Web of Life dataset) which collects the pollination networks we used. These sources can be used to obtain more exact coordinates. 


13. Suggestions to clarify explanation of regressions [[done?]]

	\begin{refquote}
		p. 7, l. 125-141 : I think the beginning of this section should be more straightforward and contrast better the information provided by M_ij and U_ij (starting with them) with that of the Jaccard index J_ij. In addition, J_ij is never mentioned again as such (but is an estimate of w_ij).
	\end{refquote}


	\textbf{R:} We have reworded this section in an attempt to clarify the relationships between $M_{ij}$, $U_{ij}$, and $J_{ij}$ and better contrast the information given by each value. We have also changed all $\omega_{ij}$ to $J_{ij}$ for simplicity. This paragraph now reads:


	\begin{quotation}
	    We calculated niche overlap for each pair of plants $i$ and $j$ by 
	    combining the number of shared interaction
	    partners ($M_{ij}$) and the number of interaction partners which are not
	    shared ($U_{ij}$) into a Jaccard index ($J_{ij}$) describing 
	    the proportion of shared interactions. Importantly, we do this while 
	    preserving the amount of information provided by each pair of plants
	    (indicated by their numbers of interaction partners). $J_{ij}$ is usually defined as: 
	    %
	    \begin{equation}
	      J_{ij} = \frac{M_{ij}}{P_i+P_j-M_{ij}} ,
	    \end{equation}
	    %
	    where $M_{ij}$ is the set of \emph{mutual} (shared) interaction partners of 
	    species $i$ and $j$ and $P_i$ and $P_j$ are the sizes of the sets of interaction 
	    \emph{partners} for species $i$ and $j$ respectively. $J_{ij}$ can also be defined:
	    %
	    \begin{equation}
	      J_{ij} = \frac{M_{ij}}{U_{ij}+M_{ij}} ,
	    \end{equation}
	    %
	    since $U_{ij}$ = $P_{i}$+$P_{j}$-2$M_{ij}$. This second definition more closely
	    relates to our methodology, which used the tuple ($M_{ij}$, $U_{ij}$) as the
	    dependent variable rather than the single value $J_{ij}$. Using this tuple form
	    allows us to give more 
	    weight to species sharing a large number of interaction partners as well as 
	    those sharing a large proportion (i.e., to emphasise pairs of generalists 
	    sharing most of their interaction partners over specialists sharing a single 
	    interaction partner). Note that species sharing a large \emph{number} of interaction partners may not share a large \emph{proportion} of interaction partners if the number of interaction partners that are not shared is also large. % Added because Josh got confused about the distinction.
	    \end{quotation}


14. R4 has identified the source of R2's confusion about whether we permuted interactions or phylogenetic distances!
	
	\begin{refquote}
		p. 14, l. 297-298 “Comparing the results in the observed networks with those obtained after permuting phylogenetic distances across pairs of plants (...)”: In the methods, interactions are shuffled, not phylogenetic distances. Doing one of the other seems equivalent for this test, but this rephrasing is a bit confusing.
	\end{refquote}


	\textbf{R:}


15. Typo in page 15, lines 322-324 [[done]]

	\begin{refquote}
		p. 15, l. 322-324 “ More closely-related pollinators did, however, tend to share fewer interaction partners (…).”: This is the first time that conservatism of interactions is checked from the pollinators perspective in the manuscript. Is it a typo or did I miss something in the manuscript?
	\end{refquote}


	\textbf{R:} Yes, this is indeed a typo. We did not have information on pollinator phylogenies and did not test anything from the pollinator perspective. We thank the Referee for pointing this out, and have corrected the line. The sentence now reads:

	Lines XX-XX:
	\begin{quotation}
		More closely-related plants in pollination networks did, however, tend to share fewer interaction partners ($\beta_{pollination}$=-0.776, $p$=0.007), similar to our within-network results above.
	\end{quotation}


16. Fig. 3, Table 2 conflict for Poaceae

	\begin{refquote}
		Legend of Fig. 3 “(…) Melastomatoaceae and Poaceae in pollination networks are not significantly different from zero.” but Table 2 shows the opposite for Poaceae.
	\end{refquote}

	\textbf{R:}


17. A confusing attempt at subtelty

	\begin{refquote}
		In the figures, \beta_distance is alternatively named “slope of regression line” (Fig. 2) and “change in log odds” (Fig. 3).
	\end{refquote}


	\textbf{R:} I think this was intentional since both interpretations apply ... but this appears to have caused some confusion. Oops.


% References

% Altermatt, F. and Pearse, I.S., 2011. Similarity and specialization of the larval versus adult diet of European butterflies and moths. The American Naturalist, 178(3), pp.372-382.

% Astegiano, J., Altermatt, F. and Massol, F., 2017. Disentangling the co-structure of multilayer interaction networks: degree distribution and module composition in two-layer bipartite networks. Scientific reports, 7(1), p.15465.

% Maynard, D.S., Serván, C.A. and Allesina, S., 2018. Network spandrels reflect ecological assembly. Ecology letters, 21(3), pp.324-334.

% Melián, C.J., Bascompte, J., Jordano, P. and Krivan, V., 2009. Diversity in a complex ecological network with two interaction types. Oikos, 118(1), pp.122-130.

% Moran, M.D., 2003. Arguments for rejecting the sequential Bonferroni in ecological studies. Oikos, 100(2), pp.403-405.

% Strauss, S.Y. and Whittall, J.B., 2006. Non-pollinator agents of selection on floral traits. Ecology and evolution of flowers, pp.120-138.

  \newpage


\bibliographystyle{prsb} 
\bibliography{abbreviated}

\end{document}