\documentclass[12pt]{letter}

% \usepackage[britdate]{canterbury-letter}
\usepackage[britdate,alyssa-signature]{canterbury-letter}
\usepackage{times}
% \usepackage{letterbib}
\usepackage{geometry}
% \usepackage[round]{natbib}
\usepackage{graphicx}
\geometry{a4paper}
\usepackage[T1]{fontenc}
\usepackage[utf8]{inputenc}
\usepackage{authblk}
\usepackage[running]{lineno}
\usepackage{amsmath,amsfonts,amssymb}
% \usepackage[margin=10pt,font=small,labelfont=bf]{caption}

%\usepackage{natbib}
% \bibpunct[; ]{(}{)}{;}{a}{,}{;}

\newenvironment{refquote}{\bigskip \begin{it}}{\end{it}\smallskip}

\newenvironment{figure}{}


\position{PhD Candidate}
\department{School of Biological Sciences}
\location{Private Bag 4800}
\telephone{+64 3 364 2729}
\fax{+64 3 364 2590}
\email{alyssa.cirtwill@pg.canterbury.ac.nz}
\url{http://stoufferlab.org}
\name{Alyssa R. Cirtwill}

% \position{PhD student}
% \department{School of Biological Sciences}
% \location{Private Bag 4800}
% \telephone{+64 3 364 2729}
% \fax{+64 3 364 2590}
% \email{alyssa.cirtwill@pg.canterbury.ac.nz}
% \url{http://stoufferlab.org}
% \name{Ms. Alyssa R. Cirtwill}


\newcommand{\mytitle}{\emph{Conservation of interaction partners between related plants varies widely across communities and between plant families.}}
\newcommand{\myjournal}{\emph{New Phytologist}}

\begin{document}

\begin{letter}{\bf Professor Alistair M. Hetherington\\
               Editor-in-Chief, New Phytologist\\
               Bailrigg House, Lancaster University\\
               Lancaster, UK\\
               LA1 4YE\\
                }

\opening{Dear Prof. Hetherington:}

We are happy to invite you to consider our manuscript 
``\mytitle'' for publication in the Interaction section of \myjournal. 
As detailed below, we believe that our study represents an important contribution
to plant science that will be of key interest to general readers of \myjournal.


\begin{enumerate}

\item What hypotheses or questions does this work address?


\smallskip 
Broadly, we test whether closely-related plants tend to share more
pollinators and/or herbivores. We test this hypothesis at the community level
and across plant families. Moreover, we link the two scales by investigating
whether the plant families in a community predict the community-level
relationship between relatedness and shared interaction partners. 
\smallskip


\item How does this work advance our current understanding of plant science?


\smallskip 
By taking a fine-scale approach to the conservation of interaction
partners, we gain a richer picture of the ways in which plant-animal
interactions are structured with respect to evolutionary history. In
particular, we demonstrate the variety of patterns of conservation across
families and propose potential mechanisms underlying these different
relationships. 
\smallskip


\item Why is this work important and timely?


\smallskip 
Plants' interactions with animals can significantly affect their
fitness for better and for worse. By comparing plant families displaying
different patterns of conservation of these interactions, we can predict which
families may experience different types of selection pressures. Thus, we can
use ecological data to guide future evolutionary research.


\end{enumerate}


We hope you will agree that the novelty of our approach and the significance
of our results justify consideration of our manuscript for \myjournal.

Thank you again for your consideration.

\closing{Regards,}


\end{letter}

% \newpage

% \setcounter{page}{1}

\end{document}
