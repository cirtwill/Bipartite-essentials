\documentclass[12pt]{article}  
\usepackage{amsmath}
\usepackage{url}
\usepackage[dvips]{graphicx}
\usepackage{multirow}
\usepackage{geometry}
\usepackage{pdflscape}
\usepackage{gensymb}
% \usepackage{rotating}
% make Figure 1 etc bold
\usepackage[labelfont=bf]{caption}
\usepackage{setspace}

\usepackage[running]{lineno}

\usepackage{dcolumn}
\newcolumntype{d}[1]{D{.}{.}{#1}}

%\usepackage{overcite}
\usepackage[round]{natbib}

\newcommand{\expect}[1]{\left\langle #1 \right\rangle}
\newcommand{\etal}{\textit{et al.\ }}

\newcommand{\beginsupplement}{%
        \setcounter{table}{0}
        \renewcommand{\thetable}{S\arabic{table}}%
        \setcounter{figure}{0}
        \renewcommand{\thefigure}{S\arabic{figure}}%
     }

% the abstract formatting
\newenvironment{sciabstract}{%
\begin{quote} \bf}
{\end{quote}}
\renewcommand\refname{References}

% margin sizes`
\topmargin 0.0cm
\oddsidemargin 0.2cm

\textwidth 16cm 
\textheight 21cm
\footskip 1.0cm


\title{Conservation of interaction partners between related plants varies widely across communities and between plant families - Supporting Information}

\author{Alyssa R. Cirtwill$^{1,2}$, Giulio V. Dalla Riva$^{3}$, Nick J. Baker$^{1}$,\\
Mikael Ohlsson$^{4}$, Isabelle Norstr\"{o}m$^{4}$, Inger-Marie Wohlfarth Hasle$^{4}$,\\
Joshua A. Thia$^{1,5}$, 
% Christie J. Webber$^{1}$, 
Daniel B. Stouffer$^{1}$}
\date{\small$^1$Centre for Integrative Ecology, School of Biological Sciences\\
\medskip$^2$Department of Ecology,\\
Environment, and Plant Sciences (DEEP)\\
Stockholm University\\
114 19 Stockholm, Sweden\\
\medskip$^3$Biomathematics Research Centre, School of Mathematics and Statistics\\
University of Canterbury\\Private Bag 4800\\
Christchurch 8140, New Zealand\\
\medskip$^4$Department of Physics, Chemistry, and Biology (IFM)\\ Link\"{o}ping University\\ 581 83 Link\"{o}ping, Sweden\\
\medskip$^5$Present Address: School of Biological Sciences\\
The University of Queensland\\Brisbane, QLD 4072, Australia }



\begin{document}
\maketitle
\baselineskip=8.5mm
\begin{spacing}{1.5}

\linenumbers
\beginsupplement
\clearpage

\section*{Supporting information 1: Sources for networks}
  \begin{table}[!h]
    \caption{Original sources for each network used in our analyses.}
    \label{sources}
    \begin{center}
    \begin{tabular}{|l l m{10cm} |}
    \hline
    Network & Network type & Original source \\
    \hline
    M\_PL\_001  & Pollination & \citep{Arroyo1982}  \\
    M\_PL\_002  & Pollination & \citep{Arroyo1982}  \\
    M\_PL\_003  & Pollination & \citep{Arroyo1982}  \\
    M\_PL\_004  & Pollination & \citep{Barrett1987} \\
    M\_PL\_005  & Pollination & \citep{Clements1923}  \\
    M\_PL\_006  & Pollination & \citep{Dicks2002} \\
    M\_PL\_007  & Pollination & \citep{Dicks2002} \\
    M\_PL\_008  & Pollination & \citep{Dupont2003}  \\
    M\_PL\_009  & Pollination & \citep{Elberling1999} \\
    M\_PL\_010  & Pollination & Elberling, H. \& Olesen, J. M. Unpublished. \\
    M\_PL\_011  & Pollination & \citep{Olesen2002a}  \\
    M\_PL\_012  & Pollination & Olesen, J. M. Unpublished.  \\
    M\_PL\_013  & Pollination & \citep{Ollerton2003}  \\
    M\_PL\_014  & Pollination & \citep{Hocking1968} \\
    M\_PL\_015  & Pollination & \citep{Petanidou1991} \\
    M\_PL\_016  & Pollination & \citep{Herrera1988} \\
    M\_PL\_017  & Pollination & \citep{Memmott2002} \\
    M\_PL\_018  & Pollination & Olesen, J. M. Unpublished.  \\
    M\_PL\_019  & Pollination & \citep{Inouye1988}  \\
    M\_PL\_020  & Pollination & \citep{Kevan1970} \\
    M\_PL\_021  & Pollination & \citep{Kato1990}  \\
    M\_PL\_022  & Pollination & \citep{Medan2002} \\
    M\_PL\_023  & Pollination & \citep{Medan2002} \\
    M\_PL\_024  & Pollination & \citep{Mosquin1967} \\
    M\_PL\_025  & Pollination & \citep{Motten1982}  \\
    M\_PL\_026  & Pollination & \citep{McMullen1993}  \\
    M\_PL\_027  & Pollination & \citep{Primack1983} \\
    M\_PL\_028  & Pollination & \citep{Primack1983} \\
    M\_PL\_029  & Pollination & \citep{Primack1983} \\
    M\_PL\_030  & Pollination & \citep{Ramirez1992} \\
    M\_PL\_031  & Pollination & \citep{Ramirez1989} \\
    \hline
    \end{tabular}
    \end{center}
    \end{table}

    \clearpage
    \newpage

    \begin{table*}[h!]
    \begin{center}
    \begin{tabular}{|l l m{6cm} |}
    \hline
    Network & Network type & Original source \\
    \hline
    M\_PL\_032  & Pollination & \citep{Schemske1978}  \\
    M\_PL\_033  & Pollination & \citep{Small1976} \\
    M\_PL\_034  & Pollination & \citep{SmithRamirez2005}  \\
    M\_PL\_035  & Pollination & \citep{Percival1974}  \\
    M\_PL\_036  & Pollination & Olesen, J. M. Unpublished.  \\
    M\_PL\_037  & Pollination & \citep{Montero2005} \\
    M\_PL\_038  & Pollination & \citep{Montero2005} \\
    M\_PL\_039  & Pollination & \citep{Stald2003} \\
    M\_PL\_040  & Pollination & \citep{Ingversen2006} \\
    M\_PL\_041  & Pollination & \citep{Ingversen2006} \\
    M\_PL\_042  & Pollination & \citep{Philipp2006} \\
    M\_PL\_043  & Pollination & \citep{Montero2005} \\
    M\_PL\_044  & Pollination & \citep{Kato2000}  \\
    M\_PL\_045  & Pollination & \citep{Lundgren2005}  \\
    M\_PL\_046  & Pollination & \citep{Bundgaard2003} \\
    M\_PL\_047  & Pollination & \citep{Dupont2009a}  \\
    M\_PL\_048  & Pollination & \citep{Dupont2009a}  \\
    M\_PL\_049  & Pollination & \citep{Bek2006}  \\
    M\_PL\_050  & Pollination & \citep{Stald2003} \\
    M\_PL\_051  & Pollination & \citep{Vazquez2002} \\
    M\_PL\_052  & Pollination & \citep{Witt1998}  \\
    M\_PL\_053  & Pollination & \citep{Yamazaki2003}  \\
    M\_PL\_054  & Pollination & \citep{Kakutani1990}  \\
    M\_PL\_055  & Pollination & \citep{Kato1996}  \\
    M\_PL\_056  & Pollination & \citep{Kato1993}  \\
    M\_PL\_057  & Pollination & \citep{Inoue1990} \\
    M\_PL\_058  & Pollination & \citep{Bartomeus2008} \\
    M\_PL\_059  & Pollination & \citep{Bezerra2009} \\
    Basset  & Herbivory & \citep{Basset1996}  \\
    Bluthgen  & Herbivory & \citep{Bluthgen2006}  \\
    Bodner  & Herbivory & \citep{Bodner2010}  \\
    Coley & Herbivory & \citep{Coley2006} \\
    Ibanez  & Herbivory & \citep{Ibanez2013}  \\
    Joern\_altuda  & Herbivory & \citep{Joern1979} \\
    Joern\_marathon  & Herbivory & \citep{Joern1979} \\
    Novotny & Herbivory & \citep{Novotny2012} \\
    Peralta & Herbivory & \citep{Peralta2016} \\
    Sheldon & Herbivory & \citep{Sheldon1978} \\
    Ueckert & Herbivory & \citep{Ueckert1971} \\
    \hline
    \end{tabular}
    \end{center}
    \end{table*}
\clearpage
\newpage

\section*{Supporting information 2: Repeating our analyses with proportion of shared partners}

    We repeated some of our analyses using the proportion of shared interaction partners as the response rather than a tuple describing the number of interaction partners that are shared and not shared. This approach is common when performing logistic regressions, but loses information about the varying weights of evidence about different plant pairs. This lost information can limit the power of the regression to detect weak trends.


    Across all networks, more distantly-related plants remained less likely
    to share interaction partners, but this relationship was not significant 
    in herbivory networks ($\beta_{distance}$=-1.07, $p$=0.477; compare to 
    $\beta_{distance}$=-6.82, $p$\textless0.001 in the main text). Plants in pollination networks again tended to share fewer interaction partners overall ($\beta_{pollination}$=-0.881, $p$\textless0.001; compare to $
    \beta_{pollination}$=-1.44, $p$\textless0.001 in the main text), and the decrease in overlap with increasing phylogenetic distance was steeper than in herbivory networks ($\beta_{distance:pollination}$=-26.8, $p$\textless0.001; compare to $\beta_{distance:pollination}$=-18.5, $p$\textless0.001 in the main text). Thus, while we observed the same trends when treating the response as a proportion and as a tuple, the tuple format displayed these trends more clearly. This is likely because of the additional informaion incorporated in the tuple.


    Within networks, using the proportion of shared interaction partners rather than a tuple resulted in significant relationships between phylogenetic distance and overlap in interaction partners in five herbivory networks and seven pollination networks (compare to seven and 34 networks, respectively, in the main text). In each case, the proportion of shared interaction partners decreased with increasing phylogenetic distance. Again, this demonstrates the greater power of a Jaccard regression with a tuple response to capture weak trends. The best-fit models for individual networks show much less variation when framing the response as a proportion (Fig.~\ref{lineplot}).


    \begin{figure}[!h]
        \begin{center}
          \centerline{\includegraphics*[width=.75\textwidth]{Figures/dataplots/proportion_regression_lines_full_color.eps}}
        \end{center}
         \caption{\small When considering niche overlap as the proportion of shared interaction partners rather than both the number of interaction partners that are shared and not shared, there is much less variability between networks. The overall trends (thicker, darker lines) observed are also much weaker than those detected when using all of the available information. Compare with Fig. 1, main text. Both figures are plotted on the same scale.
         }
        \label{lineplot}
      \end{figure}



\clearpage

\section*{Supporting information 3: Distributions of $p$-values for permuted networks}


    Comparing the permuted networks to permutations of the permuted networks, the slope obtained from the initial permuted network showed no clear relationship to the slopes obtained from 500 permutations of the permuted network. Averaged over the 1000 permutations of each observed network, the slope of the permuted network was more extreme than 48.1-51.3\% of the permutations of the permuted network. This confirms that shuffling phylogenetic distances between plant pairs destroys the relationship between distance and interaction partner overlap, and that further shuffling distances does not have a predictable effect.


    \begin{figure}[!h]
        \begin{center}
          \centerline{\includegraphics*[width=.99\textwidth]{Figures/random_p_distributions.eps}}
        \end{center}
        \vspace{-1cm}
         \caption{\small Nearly uniform distributions of $p$-values where obtained when comparing the strength of the relationship between niche overlap and phylogenetic distance in 999 permutations of each network in our dataset with 500 permutations of each permuted network. Each line in each panel represents the histogram of $p$-values for one network. Bins are 0.1 wide. Purple lines in panels A-L represent pollination networks while green lines in panels M-N represent plant-herbivore networks. A list of networks shown on each panel follows; see Table S1 for original sources.
         % Networks are as follows (darkest to lightest): \textbf{A)} M\_PL\_001, M\_PL\_002, M\_PL\_003, M\_PL\_004, M\_PL\_005;      \textbf{B)} M\_PL\_006, M\_PL\_007, M\_PL\_008, M\_PL\_009, M\_PL\_010;      \textbf{C)} M\_PL\_011, M\_PL\_012, M\_PL\_013, M\_PL\_014, M\_PL\_015;      \textbf{D)} M\_PL\_016, M\_PL\_017, M\_PL\_018, M\_PL\_019, M\_PL\_020;      \textbf{E)} M\_PL\_021, M\_PL\_022, M\_PL\_023, M\_PL\_024, M\_PL\_025;   \textbf{F)} M\_PL\_026, M\_PL\_027, M\_PL\_028, M\_PL\_029, M\_PL\_030;     \textbf{G)} M\_PL\_031, M\_PL\_032, M\_PL\_033, M\_PL\_034, M\_PL\_035;   \textbf{H)} M\_PL\_036, M\_PL\_037, M\_PL\_038, M\_PL\_039, M\_PL\_040;     \textbf{I)} M\_PL\_041, M\_PL\_042, M\_PL\_043, M\_PL\_044, M\_PL\_045; \textbf{J)} M\_PL\_046, M\_PL\_047, M\_PL\_048, M\_PL\_049, M\_PL\_050;      \textbf{K)} M\_PL\_051, M\_PL\_052, M\_PL\_053, M\_PL\_054, M\_PL\_055;     \textbf{L)} M\_PL\_056, M\_PL\_057, M\_PL\_058, M\_PL\_059;      \textbf{M)} Ibanez, Joern\_altuda, Joern\_marathon, Peralta, Sheldon;      \textbf{N)} Ueckert, Basset, Bluthgen, Bodner, Coley, Novotny.     For original sources for each network, see Table S1.
         }
        \label{within_family_regression}
      \end{figure}


    \subsection*{Networks shown in each panel of Figure S2}

    All names are as in Table S1. Networks included in each panel (from darkest to lightest line colours) are:


    \noindent \textbf{A)} M\_PL\_001, M\_PL\_002, M\_PL\_003, M\_PL\_004, M\_PL\_005;      \\
    \textbf{B)} M\_PL\_006, M\_PL\_007, M\_PL\_008, M\_PL\_009, M\_PL\_010;    \\
    \textbf{C)} M\_PL\_011, M\_PL\_012, M\_PL\_013, M\_PL\_014, M\_PL\_015;  \\
    \textbf{D)} M\_PL\_016, M\_PL\_017, M\_PL\_018, M\_PL\_019, M\_PL\_020;   \\
    \textbf{E)} M\_PL\_021, M\_PL\_022, M\_PL\_023, M\_PL\_024, M\_PL\_025;  \\
    \textbf{F)} M\_PL\_026, M\_PL\_027, M\_PL\_028, M\_PL\_029, M\_PL\_030;  \\
    \textbf{G)} M\_PL\_031, M\_PL\_032, M\_PL\_033, M\_PL\_034, M\_PL\_035; \\
    \textbf{H)} M\_PL\_036, M\_PL\_037, M\_PL\_038, M\_PL\_039, M\_PL\_040;   \\
    \textbf{I)} M\_PL\_041, M\_PL\_042, M\_PL\_043, M\_PL\_044, M\_PL\_045;\\
    \textbf{J)} M\_PL\_046, M\_PL\_047, M\_PL\_048, M\_PL\_049, M\_PL\_050;  \\
    \textbf{K)} M\_PL\_051, M\_PL\_052, M\_PL\_053, M\_PL\_054, M\_PL\_055;   \\
    \textbf{L)} M\_PL\_056, M\_PL\_057, M\_PL\_058, M\_PL\_059;  \\
    \textbf{M)} Ibanez, Joern\_altuda, Joern\_marathon, Peralta, Sheldon;  \\
    \textbf{N)} Ueckert, Basset, Bluthgen, Bodner, Coley, Novotny.

\clearpage

\section*{Supporting information 4: Details of within-family regressions}

    Models for nine families could not be fit because there was no variation in the phylogenetic distance between plants, their numbers of shared interaction partners, or both. These families were \emph{Amaranthaceae}, \emph{Araliaceae}, \emph{Cactaceae}, \emph{Cornaceae}, \emph{Gentianaceae}, \emph{Liliaceae}, \emph{Oxalidaceae}, \emph{Rhamnaceae}, and \emph{Zingiberaceae}. Further, we could not fit a model for \emph{Lauraceae} in pollination networks or \emph{Sapindaceae} in herbivory or pollination networks as only one plant pair in each network type shared any interaction partners.


    Only nine families were sufficiently well-represented to fit models for shared herbivores. Five of these were also well-represented in plant-pollinator networks: \emph{Asteraceae}, \emph{Fabaceae}, \emph{Melastomataceae}, \emph{Poaceae}, and \emph{Rubiaceae}. In total, there were 48 families which were well-represented enough to fit models for shared pollinators. Note that singular fits were obtained for \emph{Amaranthaceae}, \emph{Araliaceae}, \emph{Boraginaceae},
    \emph{Campanulaceae}, \emph{Caryophyllaceae}, \emph{Ericaceae}, \emph{Geraniaceae}, 
    \emph{Hydrangeaceae}, \emph{Malvaceae}, \emph{Oxalidaceae}, \emph{Primulaceae}, 
    \emph{Saxifragaceae}, and \emph{Verbenaceae}. 
  



\end{spacing}

\clearpage

\newpage
\bibliographystyle{newphy}
\renewcommand*{\bibfont}{\raggedright}
\bibliography{manual}

\end{document}
