\documentclass[12pt]{article}  
\usepackage{amsmath}
\usepackage{url}
\usepackage[dvips]{graphicx}
\usepackage{multirow}
\usepackage{geometry}
\usepackage{pdflscape}
\usepackage{gensymb}
% \usepackage{rotating}
% make Figure 1 etc bold
\usepackage[labelfont=bf]{caption}
\usepackage{setspace}

\usepackage[running]{lineno}

\usepackage{dcolumn}
\newcolumntype{d}[1]{D{.}{.}{#1}}

%\usepackage{overcite}
\usepackage[round]{natbib}

\newcommand{\expect}[1]{\left\langle #1 \right\rangle}
\newcommand{\etal}{\textit{et al.\ }}

\newcommand{\beginsupplement}{%
        \setcounter{table}{0}
        \renewcommand{\thetable}{S\arabic{table}}%
        \setcounter{figure}{0}
        \renewcommand{\thefigure}{S\arabic{figure}}%
     }

% the abstract formatting
\newenvironment{sciabstract}{%
\begin{quote} \bf}
{\end{quote}}
\renewcommand\refname{References}

% margin sizes`
\topmargin 0.0cm
\oddsidemargin 0.2cm

\textwidth 16cm 
\textheight 21cm
\footskip 1.0cm


\title{Conservation of interaction partners between related plants varies widely across communities and between plant families - Supporting Information}

\author{Alyssa R. Cirtwill$^{1}$, Giulio V. Dalla Riva$^{2}$, Nick J. Baker$^{1}$,\\
 Joshua A. Thia$^{1,3}$, Christie J. Webber$^{1}$, Daniel B. Stouffer$^{1}$, MSC helpers from Linkoping}
\date{\small$^1$Centre for Integrative Ecology, School of Biological Sciences\\
    \medskip$^2$Biomathematics Research Centre, School of Mathematics and Statistics\\
            University of Canterbury\\Private Bag 4800\\
Christchurch 8140, New Zealand\\
\medskip$^3$Present Address: School of Biological Sciences\\
University of Queensland\\Brisbane, QLD 4072, Australia }



\begin{document}
\maketitle
\baselineskip=8.5mm
\begin{spacing}{1.0}


Supporting information: 1 file containing 4 sections.

\section*{Supporting information 1: Sources for networks}




\end{spacing}

\newpage
\bibliographystyle{newphy}
\renewcommand*{\bibfont}{\raggedright}
\bibliography{noisn}


\newpage
\section*{Tables}

  \begin{table}[!h]
  \caption{\small Change in log odds (per million years of phylogenetic distance) of a pair of plants in the same family sharing a herbivore.}
  \label{family_slopes_ph}
  \begin{tabular}{|l  rr|}
  \hline
    Family & Change in log odds & $P$-value \\
    \hline
    \emph{Asteraceae} & -1.73 &  0.550 \\
    \emph{Euphorbiaceae} & -19.2 & \textbf{\textless0.001} \\
    \emph{Fabaceae} & 18.7 &  \textbf{0.046} \\
    \emph{Melastomataceae} & -13.2 & \textbf{0.022} \\
    \emph{Moraceae} & -2.13 & 0.092 \\
    \emph{Nothofagaceae} & -595 & \textgreater{0.999} \\
    \emph{Pinaceae} &  -25.8 & 0.733 \\
    \emph{Poaceae} & -4.50 & \textbf{0.020} \\
    \emph{Rubiaceae} & -8.16 &  \textbf{0.006} \\
  \hline
  \end{tabular}
  \smallskip
  \footnotesize

  Nine plant families were sufficiently diverse in our  dataset to permit this analysis
  (see \emph{Materials \\and Methods} for details). For each pattern of overlap, we show the change
  in log odds per million years \\and the associated $P$-value. Statistically significant values are
  indicated in bold. \\

  \end{table}

  \begin{table}[!h]
  \caption{
  \small Change in log odds (per million years of phylogenetic distance) of a pair of plants in the same family sharing a pollinator.}
  \small
  \label{family_slopes_pp}
  \begin{tabular}{|l  rr|}
    \hline
    Family  & Change in log odds & $P$-value \\
    \hline
    \emph{Adoxaceae} &  -65.8 & 0.163 \\
    \emph{Amaryllidaceae} &  -17.9 & \textbf{0.015} \\
    \emph{Apiaceae} &  10.9  & \textbf{0.006} \\
    \emph{Apocynaceae} &  -6.96  & \textbf{0.037} \\
    \emph{Asparagaceae} &  -6.23  & 0.189 \\
    \emph{Asteraceae}* &  -1.47  & \textbf{\textless0.001} \\
    \emph{Berberidaceae} &  -1.48$\times10^3$ & \textgreater0.999 \\
    \emph{Boraginaceae} &  -5.15  & \textbf{\textless0.001} \\
    \emph{Brassicaceae} &  -11.2 & 0.072 \\
    \emph{Calceolariaceae} &  156 & 0.998 \\
    \emph{Campanulaceae} &  334 & 0.999 \\
    \emph{Caprifoliaceae} &  0.310 & 0.959 \\
    \emph{Caryophyllaceae} &  2.09 & 0.644 \\
    \emph{Cistaceae} &  -11.4 & \textbf{\textless0.001} \\
    \emph{Convolvulaceae} &  -1.84  & 0.837 \\
    \emph{Ericaceae} &  4.61 & 0.116 \\
    \emph{Fabaceae}* &  -12.9 & \textbf{\textless0.001} \\
    \emph{Geraniaceae} &  -3.31  & 0.624 \\
    \emph{Hydrangeaceae} &  0.057 & 0.982 \\
    \emph{Iridaceae} &  -27.9 & 0.078 \\
    \emph{Lamiaceae} &  -5.01  & \textbf{\textless0.001} \\
    \emph{Loasaceae} &  -865  & \textgreater0.999 \\
    \emph{Malpighiaceae} &  2.80 & 0.168 \\
    \emph{Malvaceae} &  -5.56  & 0.363 \\
    \emph{Melastomataceae}* &  5.19 & 0.577 \\
    \emph{Montiaceae} &  -1.12  & 0.870 \\
    \emph{Myrtaceae} &  8.55 & 0.071 \\
    \emph{Oleaceae} &  0.995 & 0.855 \\
    \emph{Onagraceae} &  -556  & \textgreater0.999 \\
    \emph{Orchidaceae} &  -14.5 & 0.145 \\
    \emph{Orobanchaceae} &  24.2  & 0.326 \\
    \emph{Papaveraceae} &  -11.2 & 0.511 \\
    \emph{Phyllanthaceae} &  9.99 & 0.433 \\
    \emph{Plantaginaceae} &  -8.48  & \textbf{0.001} \\
    \emph{Poaceae}* &  69.2  & \textbf{0.003} \\
    \emph{Polygonaceae} &  -14.8 & \textbf{\textless0.001} \\
    \emph{Primulaceae} &  14.9  & 0.343 \\
    \emph{Ranunculaceae} &  -38.0 & \textbf{\textless0.001} \\
    \emph{Rosaceae} &  0.759 & 0.735 \\
    \emph{Rubiaceae}* &  -13.0 & \textbf{0.026} \\
    \emph{Salicaceae} &  -1.90  & 0.545 \\
    \emph{Sapindaceae} &  821 & 0.999 \\
    \emph{Saxifragaceae} &  -0.092  & 0.992 \\
    \emph{Solanaceae} &  -21.9 & 0.189 \\
    \emph{Tropaeolaceae} &  192 & 0.997 \\
    \emph{Verbenaceae} &  -9.03  & 0.627 \\
    \emph{Violaceae} &  -0.487  & 0.974 \\
  \hline
  \end{tabular}
  \smallskip
  \footnotesize

    We were able to fit these models to 47 plant families (see \emph{Materials and Methods} for     details). 
    Families \\ marked with an asterisk were also sufficiently diverse in herbivory networks. 
    Statistically significant values \\ are indicated in bold. 

    \end{table}

\clearpage

\end{document}
