\documentclass[12pt]{article}  
\usepackage{amsmath}
\usepackage{url}
\usepackage[dvips]{graphicx}
\usepackage{multirow}
\usepackage{geometry}
\usepackage{pdflscape}
\usepackage{gensymb}
% \usepackage{rotating}
% make Figure 1 etc bold
\usepackage[labelfont=bf]{caption}
\usepackage{setspace}

\usepackage[running]{lineno}

\usepackage{dcolumn}
\newcolumntype{d}[1]{D{.}{.}{#1}}

%\usepackage{overcite}
\usepackage[round]{natbib}

\newcommand{\expect}[1]{\left\langle #1 \right\rangle}
\newcommand{\etal}{\textit{et al.\ }}

\newcommand{\beginsupplement}{%
        \setcounter{table}{0}
        \renewcommand{\thetable}{S\arabic{table}}%
        \setcounter{figure}{0}
        \renewcommand{\thefigure}{S\arabic{figure}}%
     }

% the abstract formatting
\newenvironment{sciabstract}{%
\begin{quote} \bf}
{\end{quote}}
\renewcommand\refname{References}

% margin sizes`
\topmargin 0.0cm
\oddsidemargin 0.2cm

\textwidth 16cm 
\textheight 21cm
\footskip 1.0cm


\title{Conservation of interaction partners between related plants varies widely across communities and between plant families.}

\author{Alyssa R. Cirtwill$^{1}$, Giulio V. Dalla Riva$^{2}$, Nick J. Baker$^{1}$,\\
 Joshua A. Thia$^{1,3}$, Christie J. Webber$^{1}$, Daniel B. Stouffer$^{1}$, MSC helpers from Linkoping}
\date{\small$^1$Centre for Integrative Ecology, School of Biological Sciences\\
    \medskip$^2$Biomathematics Research Centre, School of Mathematics and Statistics\\
            University of Canterbury\\Private Bag 4800\\
Christchurch 8140, New Zealand\\
\medskip$^3$Present Address: School of Biological Sciences\\
University of Queensland\\Brisbane, QLD 4072, Australia }



\begin{document}
\maketitle
\baselineskip=8.5mm
\begin{spacing}{1.0}

\section*{Word Counts}

Main text: 4522

\begin{itemize}
  \item Introduction: 805 
  \item Materials and Methods: 1831 
  \item Results: 665
  \item Discussion: 1139 
  \item Acknowledgements: 83
\end{itemize}



Figures: 3


Tables: 2


Supporting information: 1 file containing 4 sections.


\vspace{0.4 in}

\section*{Summary}

  \begin{itemize}
    \item Related plants are often hypothesised to interact with similar sets of 
          pollinators and herbivores, but empirical support for this idea is mixed.
          Here we argue that this may be because some plant families vary in their 
          tendency to share interaction partners.

    \item We introduce a novel approach with which to quantify
          overlap of interaction
          partners for each pair of plants in 59 pollination and 11 herbivory
          networks.  We then tested for relationships between phylogenetic 
          distance and overlap within each network, and whether these 
          relationships varied with the composition of the plant community.
          Finally, we tested for different relationships within well-represented
          plant families.

    \item Across all networks, more closely-related plants tended to have 
          greater overlap, and this tendency was stronger in herbivory 
          networks than pollination networks. These relationships were also significantly related 
          to the composition of the network's plant component. Within plant families, 
          relationships varied greatly in both network types. 

    \item The variety of relationships between phylogenetic distance and
          interaction partners in different plant families likely
          reflects a variety of ecological and evolutionary processes. To 
          understand the distribution of interactions within a community, 
          it is therefore important to consider factors affecting particular 
          plant families.

  \end{itemize}


\section*{Keywords}

defensive syndrome, ecological networks, herbivory, niche overlap, phylogenetic signal, pollination, pollination syndrome, specialisation

\end{spacing}
\begin{spacing}{1.5}

\section*{Introduction}
\linenumbers

  Interactions with animals affect plants' life cycles in several critical
  ways~\citep{Mayr2001,Sauve2015}. On one hand,
  pollination and other mutualistic interactions contribute
  to the reproductive success of many angiosperms~\citep{Ollerton2011}. 
  On the other, herbivores consume plant tissues~\citep{McCall2006} which
  costs plants energy and likely lowers their fitness.
  In both cases, these interactions do not occur randomly but
  are strongly influenced by plants' phenotypes. For example, plants that produce
  abundant or high-quality nectar may receive more visits from
  pollinators~\citep{Robertson1999} whereas plants that produce noxious secondary metabolites
  may suffer fewer herbivores~\citep{Johnson2014}. A plant's traits are also 
  likely to determine which specific pollinators and herbivores interact with that plant.
  Plants with different defences (e.g., thorns vs. chemical defences) may deter 
  different groups of herbivores~\citep{Ehrlich1964,Johnson2014}, and the concept
  of pollination syndromes has often been used to group plants into phenotypic
  classes believed to attract certain groups of pollinators~\citep{Waser1996,Fenster2004,Ollerton2009}.


  [[update refs, add definition of plant roles]]
  If attractive and/or defensive traits are heritable,
  then we can reasonably expect that related plants will have similar 
  patterns of interactions with animals~\citep{Schemske1999}.
  Recent studies that have investigated this question at the level of whole
  communities, however, have yielded mixed results~\citep{Rezende2007a,Gomez2010,Rohr2014a,Fontaine2015,Lind2015}.
  In particular, significant phylogenetic signal in plants' interaction partners
  ---the tendency for more closely-related plants to have more similar interactions---
  tends to be rare in empirical networks (\citealp{Rezende2007a,Lind2015}; but see~\citealp{Elias2013,Fontaine2015}).
  Further, plants' roles [[how defined?]] within networks tend to be less phylogenetically constrained than those of 
  animals~\citep{Rezende2007a,Chamberlain2014,Rohr2014,Vamosi2014,Lind2015}. [[GVDR is skeptical of this - check partiteness, size, type of role and add more detail to support.]]


  Several mechanisms that might weaken the conservation of interactions
  have been identified in the literature. Pollination and herbivory may be affected
  by a wide variety of traits, and not all of these are likely to be
  phylogenetically conserved~\citep{Rezende2007,Kursar2009}. If, for example,
  floral displays are strongly affected by environmental conditions~\citep{Canto2004}, 
  then pollinators may not be predicted by plants' phylogenies.
  Even if the traits affecting pollination and herbivory are
  heritable, plants may experience conflicting selection pressures that
  weaken the overall association between plant phylogeny and interaction
  partners~\citep{Armbruster1997,Lankau2007,Siepielski2010,Wise2013}. 
  For instance, floral traits that
  are attractive to pollinators can also increase 
  herbivory~\citep{Strauss2002,Adler2004,Theis2006}. 
  Conversely, herbivory can reduce pollination by inducing chemical 
  defences~\citep{Adler2006} or altering floral display or nectar 
  availability~\citep{Strauss1997}. Observed patterns
  of similarity in plants' interaction partners therefore represent
  a mixture of environmental effects and various selection pressures as
  well as plants' shared phylogenetic history.


  A further complication is the possibility that the relationship between
  plants' relatedness and the similarity of their interaction partners is
  not constant across plant clades. Closely-related plants
  in one clade might be under strong selection to favour dissimilar
  sets of pollinators to avoid exchanging pollen with other
  species~\citep{Levin1970,Bell2005,Mitchell2009}. Similar pressures 
  could also affect related plants' defences against herbivores if 
  congeners tend to grow in the same places such that herbivores 
  could easily move between them.  Unrelated plants might also 
  converge upon similar phenotypes, attracting a particularly 
  efficient or abundant pollinator~\citep{Ollerton1996,Ollerton2009}. 
  Likewise, herbivores may be able to depredate sets of unrelated 
  plants if the plants have evolved similar defences~\citep{Pichersky2000}. 
  In either case, dissimilarity of interactions among related species 
  or similarity of interactions among unrelated species could result 
  in weaker phylogenetic signal across an entire plant community. 
  Moreover, all of the aforementioned hypotheses 
  are non-exclusive; different processes likely affect different
  clades, and these processes might be associated with different 
  pressures imposed by pollination and herbivory. 


  Here we use a novel approach to investigate how overlap in 
  interaction partners between pairs of plants (henceforth ``niche 
  overlap'') varies over phylogenetic distance. Whereas previous 
  studies have focused on the presence or absence of phylogenetic
  signal across entire networks, we are able to investigate the
  relationship between niche overlap and phylogenetic distance in
  within networks as well as different plant families. Specifically,
  we test whether niche overlap decreases over increasing phylogenetic
  distance in a large dataset of pollination and herbivory networks, 
  whether the plant family composition of a community affects the
  relationship between niche overlap and phylogenetic distance in that 
  community, and whether the relationship between niche overlap and 
  phylogenetic distance differs across plant families.


\section*{Materials and Methods}

\subsection*{Network data}

  We studied tested for phylogenetic signal in niche overlap within a 
  set of 59 pollination and 11 herbivory networks. These networks span 
  a range of biomes (desert to scrub forest to grassland) and 
  countries (Sweden to Australia). The herbivory networks included a 
  variety of types of herbivores but were dominated by insects 
  consuming leaves. To ensure that we were analysing interactions 
  influenced by similar sets of traits across networks, we restricted 
  our herbivory networks to insects consuming leaves and excluded 
  sap-sucking, leaf-mining, and galling insects as well as seed 
  predators and xylophagous insects; all of these interactions involve 
  different plant tissues and means of feeding than leaf consumption 
  and so may be influenced by different plant and insect traits. 
  Specifically, we removed any non-leaf consuming insects and any 
  plants which had no interaction partners after removing other types  
  of herbivores. The adjusted networks range in size between 19 and 
  997 total species (mean=162, median=97) with  between 8 and 132 
  plant species (mean=39.1, median=29.5). See \emph{Table S1, 
  Supporting Information} [[NOT DONE!]]for details on the original sources of all 
  networks. 

\subsection*{Phylogenetic data}

  In order to fit the plant species in all networks to a common phylogeny, 
  we first compared all species and genus names with the 
  National Center for Biotechnology Information
  and Taxonomic Name Resolution Service databases to ensure
  correctness. This was done using the function `get\_tsn' in the R~\citep{R}
  package taxize~\citep{taxize1,taxize2}. Species which could not 
  be assigned to an accepted taxonomic name (e.g., `Unknown Forb') were 
  discarded, as were those with non-unique common names and no binomial 
  name given (e.g., `Ragwort) or binomial names that could not be definitively 
  linked to  higher taxa (e.g., \emph{`Salpiglossus sp.'}). We were left with 
  2341 unique species in 1027 genera and 195 families. On average, 11.43\% of 
  plants were removed from each network (median 4.60\%, range 0-55.10\%).


  We then estimated phylogenetic distances between species. To accomplish 
  this, we constructed a phylogenetic tree for our dataset based on a dated
  `mega-tree' of angiosperms~\citep{Zanne2014}. Some species were not included
  in the angiosperm mega-tree (largely ferns and tree ferns). For angiosperms,
  a sister taxon was identified using~\citet{APW} and the species added manually.
  Ferns, tree ferns, and a single club moss were added to the base of the tree.
  This means that closely-related non-angiosperm species appear to have very long 
  phylogenetic distances between them. For this reason, we excluded comparisons 
  between pairs of non-angiosperms from our analyses. As only two networks (both 
  plant-herbivore networks) included more than one such species and non-angiosperms
  were always a small minority of any network, we do not 
  believe that omitting these comparisons has greatly affected our results.
  To obtain trees for each network, we 
  pruned the dated mega-tree to include only species in that network.


\subsection*{Calculating niche overlap within communities}

  We calculated niche overlap for each pair of species within a community 
  using a Jaccard index to describe the number of shared interaction 
  partners, augmented with the number of interaction partners which were 
  not shared. The Jaccard index $J_{ij}$ describes the proportion of 
  shared interaction partners for for species $i$ and $j$ and is defined as: 

  \begin{equation}
    J_{ij} = \frac{M_{ij}}{P_i+P_j-M_{ij}} ,
  \end{equation}

  where $M_{ij}$ is the set of \emph{mutual} (shared) interaction partners of 
  species $i$ and $j$ and $P_i$ and $P_j$ are the sizes of the sets of interaction 
  \emph{partners} for species $i$ and $j$ respectively. We wished to give more 
  weight to species sharing a large number of interaction partners as well as 
  those sharing a large proportion (i.e., to emphasize pairs of generalists 
  sharing most of their interaction partners over specialists sharing a single 
  interaction partner). We therefore recorded, for each species, the number of 
  shared interaction partners $M_{ij}$ and the number of interaction partners 
  for each species pair that were not shared 
  ($U_{ij}$ = $P_{i}$+$P_{j}$-2$M_{ij}$). Instead of a single index $J_{ij}$ 
  we therefore keep track of the information needed to compute
  niche between species $i$ and $j$ as a tuple: ($M_{ij}$, $U_{ij}$).


  % To fully describe the extent to which two plants' niches overlap,
  % we defined the overlap between two plants' sets of interaction partners
  % by recording the frequencies with which pairs of animals (where each animal 
  % interacted with at least one plant) fall into three unique patterns
  % (Fig.~\ref{overlap_patterns}). In the first pattern, both plants interact 
  % with both animals, indicating total overlap for that quartet. In the second pattern, one plant
  % interacts with both animal partners while the other
  % interacts with only one animal, indicating partial overlap. In the third pattern,
  % each animal interacts with only one plant, indicating no overlap. 
  % Taken together, the frequencies of these three patterns of
  % overlap can be used to describe the degree to which two plants have
  % similar interaction partners. %


  % Using the three patterns defined above provides more detail than other measures of overlap,
  % such as the proportion of one species' partners that are shared with
  % another as given by Jaccard similarity. In particular, comparing the probability of observing each pattern
  % rather than one of the other two provides a measure of indirect
  % interactions between plants by considering pairs of animal partners rather
  % than each animal separately. For example, a pair of plants which share 
  % two interaction partners are more likely to influence each other via these partners
  % than two plants which do not share interaction partners. Moreover, our measure
  % of overlap has greater statistical power than Jaccard dissimilarity because
  % it includes information on the \emph{number} of shared interaction partners as
  % well as the proportion. For instance, a pair of plants which together interact
  % with 100 animals provides more information about shared overlap than a pair of
  % plants which together interact with only one animal whereas the Jaccard similarity
  % of both would simply be one.

  % [[It appears we switched to Jaccard and Sorenson dissimilarities - number of shared partners and number of shared partners*2/notshared.]]


\subsection*{Statistical analysis} 

  We modelled the relationship between niche overlap and phylogenetic distance
  using a logistic regression. We used both the numbers of shared ($M_{ij}$)and 
  non-shared ($U_{ij}$) partners as dependent variables and centred, scaled phylogenetic 
  distance as the independent variable. This approach is conceptually similar to 
  modelling successes and failures in a binomial-distributed process. 
  Accordingly, we assumed a binomial distribution for residuals. Regressions of
  niche overlap and phylogenetic distance within each network were fit using the
  R~\citep{R} base function ``glm''. These separate regressions avoid the potential
  for confounding the effects of different relationships in different networks. As
  we also wished to evaluate the overall trend across networks, we fit an additional
  regression of niche overlap and phylogenetic distance across all network types.
  As well as the fixed effect of phylogenetic distance, this regression included 
  fixed effects of network type (pollination or herbivory) and the interaction 
  between phylogenetic network type and random intercepts and slopes per network.
  This expanded regression was fit using the R~\citep{R} function ``glmer'' from
  package \emph{lme4}~\citep{lme4}. [[Confirm that I actually use the separate 
  regressions and not just the full one. ]]


  Note that pairs of plants are not always independent: the same plant will
  appear in many pairs. This violates the assumption of independence used when 
  calculating the significance of logistic regressions within the R~\citep{R} 
  base package or the package \emph{lme4}~\citep{lme4}. To calculate significance 
  of the regression coefficients we observed, it was therefore necessary to compare the 
  observed relationships to those in a suite of appropriately permuted networks. To create 
  these networks, we shuffled interactions among species while preserving row 
  and column totals. That is, each species retained the same number of 
  interaction partners as in the observed network but the exact set of 
  partners (and therefore niche overlaps with all other species) varied across 
  permuted networks. We preserved the observed phylogenetic relationships 
  between species in all cases. For each observed network, we created 999 such 
  permuted networks and calculated the relationship between niche overlap and 
  phylogenetic distance. This gave us a null distribution for each observed 
  network with which to determine the significance of the observed 
  relationship.


  This permutation approach also allows us to estimate type I and type II 
  error for our analysis. To do this, we created 500 permutations of each 
  permuted network and, again keeping the observed phylogenetic distances 
  between plant species, repeated our analyses. We can then determine the 
  number of permuted networks which appear to have significant 
  overlap-phylogenetic distance relationships relative to the permutations 
  of these permuted networks (type I error). [[How does this get at type II error?]] 


  % To determine how overlap of interaction partners
  % breaks down over phylogenetic distance,
  % we modelled the probabilities of observing each pattern
  % of overlap relative to the other two patterns.
  % We expected that the frequency of the high- and moderate-overlap 
  % patterns would decrease with increasing phylogenetic distance
  % between two plants while the frequency of the low-overlap pattern would
  % increase. As we expect pollination and herbivory networks could 
  % show different patterns of overlap, we included effects of network 
  % type and the interaction between network type and distance. Lastly, to
  % account for the possibility that different communities show different
  % characteristic relationships, we also included random effects of network ID on the slope 
  % and intercept, giving a mixed-effects logistic regression of the form
  % \begin{equation}
  % logit(\omega_{pnij}) \propto \delta_{ij} + \rho_{n} + \delta_{ij}\rho_{n} + N_{n} + \delta_{ij}N_{n} ,
  % \label{networklevel}
  % \end{equation}

  % \noindent where $\omega_{pnij}$ is the probability of overlap pattern $p$ occurring between
  % species $i$ and $j$ in network $n$, $\delta_{ij}$ is the phylogenetic distance between 
  % plants $i$ and $j$, $\rho_{n}$ is the network type (one in pollination networks,
  % zero in herbivory networks), and $N_n$ and $\delta_{ij}N_{n}$ are random slope and intercepts 
  % for network $n$. All models were fit using R function glmer from package lme4~\citep{lme4}.
  % Sample size for these models was the sum (over all pairs of plants) of the number of pairs 
  % of animals where each plant and each animal has at least one interaction partner. Over all 
  % networks, there were 43,288,090 such sets of plants and animals, with a median of 72 (mean 
  % 671 +/- 2247) pairs of animals per pair of plants and median 58,528 (mean 636,590)
  % plant-animal sets per network.



\subsection*{Linking network-level trends and community composition}

  Next, we examined the connection between our network-level observations
  and the plant families present in each community.
  Specifically, we tested the hypothesis that
  varying relationships between phylogenetic distance and
  pairwise niche overlap are due to the different distributions 
  of families across networks. To do this, we performed a non-parametric
  permutational multi-variate analysis of variance
  (PERMANOVA;~\citealp{Anderson2001}) using the change in log
  odds of two plants sharing an interaction partner to 
  predict the Bray-Curtis dissimilarity of networks
  based on the composition of their plant component
  (defined as the proportions of the plant community
  made up by each plant family present in the entire dataset).
  We used Bray-Curtis dissimilarity because, for a given
  pair of networks, only those plant families that appear
  in at least one network are considered~\citep{Anderson2001,Cirtwill2015}; 
  that is, the absence of rare plant families will not make 
  two networks appear more similar than they actually are. 


  Note that a PERMANOVA does not assume that the data are 
  normally distributed, but rather compares the pseudo-$F$ 
  statistic calculated from the observed data to a null 
  distribution obtained by permuting the raw data. As 
  pollination and herbivory networks might have different
  community composition, we stratified these permutations
  by network type. That is, the change in log odds for a pollination
  network could only be exchanged for that of another pollination
  network. Stratifying the permutations in this way ensures that 
  the null distribution used to calculate the $P$-value is not 
  biased by including combinations of changes in log odds and 
  community composition that would not occur because of inherent 
  differences in the two network types (e.g., \emph{Pinaceae} 
  only appeared in herbivory networks and should not be assigned 
  to pollination networks). We used 9999 such stratified permutations 
  to obtain the null distribution and obtain a $P$-value.


  The PERMANOVA tests whether there is an association between
  community composition and network-level patterns but does not
  give any information on \emph{which} plant families have the
  greatest effects. To address this, we supplemented the 
  PERMANOVA with three constrained correspondence analyses (CCAs)
  which placed plant families along an axis representing the
  change in log odds of sharing an interaction partner.
  A correspondence 
  analysis (CA) is similar to other multivariate
  analyses such as principal components analysis in that it
  reduces multivariate data to a set of orthogonal axes. A
  subset of axes that explain the majority of variation in 
  the data can then be interpreted to elucidate trends that
  were difficult to interpret in the full multivariate space.
  A constrained correspondence analysis (CCA) creates an extra
  axis based on some constraint - in this case, the change in
  log odds of sharing an interaction partner. 


\subsection*{Calculating niche overlap within families}

  Finally, we wished to compare the breakdown of overlap of interactions in
  different plant families. To do this, we used the same definitions of overlap
  and phylogenetic distance as in the within-network analysis but restricted
  our regressions to pairs of plants from the same family and the same network.
  Unlike in our previous analysis, we analysed data from pollination and
  herbivory networks separately as most well-represented plant families appeared
  in only one network type. For those families which appeared in both network types, 
  we ran separate analyses on each subset of data.


  For each plant family, within each network type, we then fit one of two similar 
  sets of models. First, when family $f$ was found in several networks of network
  type $t$, we fit mixed-effects logistic regressions for the probability of sharing an interaction partner $\omega_{ntfij}$ of the form
  \begin{equation}
    logit(\omega_{ntfij}) \propto \delta_{ij} + N_{n} ,
    \label{full}
  \end{equation}

  \noindent where $\omega_{tnfij}$ is the probability of sharing a partner 
  in network $n$ of network type $t$ for plants $i$ and $j$ in plant family $f$,
  $\delta_{ij}$ is the phylogenetic distance between plants $i$ and $j$, and
  $N_{n}$ is a random effect of network $n$.
  Second, if a plant family was represented in only one network 
  and therefore necessarily in only one network type, we omitted the 
  network-level random effect giving mixed-effects logistic regressions of the form
 \begin{equation}
    logit(\omega_{pntfij}) \propto \delta_{ij} .
    \label{minimal}
  \end{equation}

  \noindent We fit Eq.~\ref{full} using the function `glmer' from the
  R package lme4~\citep{lme4} and fit Eq.~\ref{minimal} in R using the
  function `glm' from the same package.


\section*{Results}


  \subsection*{Within-network conservation of niche overlap}

    Across all networks, overlap of interaction partners decreased with increasing phylogenetic distance and the decrease in overlap of pollinators was steeper than that in herbivores ($\beta_{distance}$=-6.82, $p$\textless0.001 and $\beta_{distance:PP}$=-18.5, $p$\textless0.001, respectively). That is, a pair of plants in the same genus was more likely to share interaction partners than a pair of plants in the same family in both types of networks, but a pair of congeners would be less likely to share pollinators than to share herbivores. Regardless of phylogenetic distance, a pair of plants were also less likely to share pollinators than herbivores ($\beta_{PP}$=-1.44, $p$\textless0.001). This may be due to the greater proportion of
    specialist pollinators than specialist herbivores. In our dataset, an
    average of 48\% (+/- 14) of pollinators in a given web were extreme 
    specialists (i.e., visited only one plant species) compared to 29\% 
    (+/- 29) of herbivores ($z$=5.62, df=68, $P$\textless0.001 
    for a binomial regression of specialists and generalists over network
    type). % Are the above betas un-scaled?
 

    Despite these general trends, there was substantial variation between 
    pollination networks, with overlap of interaction partners decreasing 
    with increasing phylogenetic distance in some networks and increasing in 
    others (Fig.~\ref{within_network_regression}). Overlap of interaction partners decreased significantly with increasing phylogenetic distance in 7/11 plant-herbivore networks and 33/59 plant-pollinator networks. In the remaining four plant-herbivore networks and 25 of the 26 remaining plant-pollinator networks, overlap of interaction partners was not related to phylogenetic distance. Overlap of interaction partners increased with increasing phylogenetic distance in only a single plant-pollinator network.% (M_PL_026).


    [[Reviewers demanded this - it seems very boring]]
    Comparing the results in the observed networks to those obtained after permuting phylogenetic distances across pairs of plants, the observed slope of the relationship between phylogenetic distance and interaction partner overlap was always more extreme (i.e., always lesser or always greater) than that obtained in the permuted networks (Fig.~\ref{obs_vs_random}). Observed networks with a negative relationship between phylogenetic distance and overlap always had a more negative slope than that obtained from the permuted networks, while the 10 networks with positive relationships between phylogenetic distance and overlap always had more positive relationships than the permuted networks.


    Comparing the permuted networks to permutations of the permuted networks, the slope obtained from the initial permuted network was quite variable with respect to the slopes obtained from 500 permutationss of the permuted network. Averaged over the 1000 permutations of each observed network, the slope of the permuted network was more extreme than 48.1-51.3\% of the permutations of the permuted network. This confirms that shuffling phylogenetic distances between plant pairs destroys the relationship between distance and interaction partner overlap, and that further shuffling distances does not have a predictable effect.


  \subsection*{Linking network-level trends and community composition} 

    We were interested in whether the slope of the relationship between phylogenetic distance varied with community composition. In a PERMANOVA of slope against community composition, stratified by network type, we did not find a significant relationship between slope and community composition ($F_{1,68}$=1.06, $p$=0.493). Of the 200 families in our dataset, only 29 were represented by more than 20 species. Lumping all other families into an "other" category and repeating the PERMANOVA, we still did not find a significant relationship between slope and community composition ($F_{1,68}$=1.12, $p$=0.409). 
    % Also performed a CCA to see which fams had most positive/negative associations, but with non-sig results I don't think it's interesting.


  \subsection*{Within-family conservation of niche overlap} [[Still of interest given negative result above?]]

    Taking all families together, the probability of species in the same family sharing interaction partners decreased wiht increasing phylogenetic distance in both plant-pollinator and plant-herbivore networks and the strength of this relationship did not differ significantly between network types ($\beta_{distance}$=-4.75, $p$=0.012; $\beta_{distance:PH}$=-1.73, $p$=0.681). Plants in plant-herbivore networks had a higher probability of sharing interaction partners with a member of the same family ($\beta_{PH}$=0.78, $p$=0.007).


    The relationship between within-family niche overlap and phylogenetic distance
    varied widely in both pollination and herbivory networks. 
    % Gotta separate plant-herb and plant-poll networks.


    For the families
    that were well represented in pollination networks, overlap decreased
    with increasing phylogenetic distance in 18 (Table~\ref{family_slopes_pp}).
    There was no significant relationship between overlap and phylogenetic distance in
    a further 15 plant families. Finally, the overlap between pairs of \emph{Polygonaceae}
    increased with increasing phylogenetic distance. 
    Of the seven plant families that were sufficiently well represented in herbivory 
    networks, overlap decreased with increasing phylogenetic distance in four 
    (Table~\ref{family_slopes_ph}; Fig.~\ref{within_family_regression}). Two
    families did not show significant relationships between phylogenetic distance and overlap,
    and in one family, \emph{Fabaceae}, overlap of interaction partners increased with 
    increasing phylogenetic distance.


    Models for two families did not converge. In both the \emph{Lauraceae}, (represented by four species in one pollination network) and the \emph{Sapindaceae} in plant-herbivore networks (represented by five species in one network), one pair of species shared a single interaction partner while all other pairs did not share any interaction partners. The model for \emph{Sapindaceae} in plant-pollinator networks did converge.


\section*{Discussion} [[not updated post new results]]

  We found broad support for the hypothesis that more
  closely-related pairs of plants have a higher degree
  of niche overlap. Using a novel method which considers
  all pairs of plants together, 
  the probability of two plants sharing the same animal 
  interaction partners decreased with increasing 
  phylogenetic distance. Considering networks separately,
  $\approx$78\%  of the pollination and all of the 
  herbivory networks exhibited the expected trend of decreasing 
  overlap with increasing distance. 


  Within families, however, there was much greater variability. 
  More than half of the plant families in each network type
  behaved as we hypothesised, with more 
  closely-related plants having greater niche overlap than 
  distantly related plants. This relationship between overlap and 
  phylogenetic distance is consistent with the idea that traits affecting 
  interactions are heritable~\citep{Schemske1999} and changing gradually
  such that closely related plants resemble their common ancestor--- and
  each other ---more than they do distantly related plants. In some families, 
  such as \emph{Asteraceae} in pollination networks, the positive slope of 
  this relationship was very shallow while in others, such as 
  \emph{Melastomataceae} in herbivory networks, the positive slope was 
  extremely steep. This could indicate different rates of phenotypic drift 
  or evolution in different families.


  In contrast, \emph{Polygonaceae} in pollination networks and 
  \emph{Fabaceae} in herbivory networks showed the opposite 
  pattern. In these families, closely-related plants had 
  \emph{lower} overlap than more distantly-related pairs of plants. 
  There are several possible reasons a plant family might
  display this pattern. First, part of the family may have recently 
  undergone a period of rapid diversification with closely-related species 
  developing novel phenotypes that attract different animal 
  interaction partners~\citep{Linder2008,Breitkopf2015}. It is also possible that the
  animals have undergone an adaptive radiation to 
  specialise on their most profitable partner~\citep{Janz2006}. Second, this pattern 
  could be the result of ecological or environmental filtering~\citep{Mayfield2009,Ackerly2010}. 
  Closely-related species which have high degrees of overlap in their interaction 
  partners might compete too severely to coexist. This is especially likely
  for plants sharing pollinators, where the loss of pollen to related species 
  might severely limit reproductive success~\citep{Levin1970,Bell2005,Mitchell2009}.
  Indeed, animal pollination and seed dispersal have been shown to act
  as filters for several plant clades~\citep{Mayfield2009}.
  Distantly-related plants with similar interaction partners,
  on the other hand, may differ in some other aspect of their niches
  and so be able to coexist. Plants sharing herbivores are unlikely to
  compete for these interaction partners, but the presence of both
  plants in a community could support higher herbivore populations
  than could one species alone~\citep{Russell2007}. If the plants 
  compete for some other resource, the combined impact of herbivory
  and competition could eliminate the rarer plant species. Distantly-related
  plants sharing herbivores, conversely, would be less likely to compete
  for vital resources and more likely to persist.


  The remaining families did not show significant relationships 
  in either direction. That is, the niche overlap between two 
  plants did not vary linearly over phylogenetic distance. Once again, there 
  are several possible drivers for this trend (or lack thereof). These plants might 
  be highly specialised on different interaction partners and therefore
  have low overlap at all levels of relatedness. In other plant families
  with more moderate levels of specialisation, it is possible 
  that pollination and/or herbivory do not exert large
  selection pressures on the plants. If traits affecting pollination
  or herbivory are not heritable in these groups~\citep{Kursar2009}
  and that their phenotypes are constrained by other factors (e.g., 
  drought tolerance), then we should not expect a relationship 
  between phylogenetic distance and overlap of interaction partners.
  Alternatively, pollination and/or herbivory might exert large 
  pressures that maintain the clade within a pollination or 
  defensive syndrome. These syndromes are commonly believed to
  predict the pollinators or herbivores with which a plant will 
  interact~\citep{Waser1996,Fenster2004,Ollerton2009,Johnson2014}.
  As some recent studies have suggested that pollination syndromes
  do not accurately predict plants' visitors in all plant 
  families~\citep{Ollerton2009},
  it may be of interest for future researchers to test whether 
  syndromes are better predictors in families with weak 
  relationships between overlap and phylogenetic distance.
  Lastly, it is possible that the absence of a linear relationship 
  between niche overlap and phylogenetic distance is because the data
  actually exhibit a strongly non-linear one. This could result, for 
  example, from an early burst of diversification followed by a period
  of stasis~\citep{Davis2014}.


  For those families which were well-represented in both pollination
  and herbivory networks, we can also contrast the 
  trends in the two network types. In all five such cases, there was
  a significant relationship between overlap and phylogenetic 
  distance in only one network type (counting the singular 
  relationships in \emph{Rubiaceae} in pollination networks as 
  non-significant). This may indicate that one type of interaction
  places greater constraints upon these families than the other. Plants
  may not be able to respond to selection on both types of interaction
  simultaneously because traits affecting pollination can also affect herbivory, and vice
  versa~\citep{Strauss1997,Strauss2002,Adler2004,Adler2006,Theis2006}.
  Associations with pollinators and herbivores may also be constrained by the
  larger structure of the community. In one recent study, plants which are 
  visited by many pollinators are also consumed by many herbivores~\citep{Sauve2015}.
  This may be because pairing antagonistic and mutualistic interactions
  balances the indirect effects of these interactions, leading to a
  more stable community~\citep{Sauve2014}. As more networks describing 
  pollination and herbivory in the same community become available, it 
  will be interesting to test this hypothesis more thoroughly.


  Altogether, our study has revealed a wide variety of relationships 
  between overlap of interaction partners and phylogenetic distance 
  between plants in the same family. Regardless of the precise mechanisms
  behind these relationships, it is clear
  that the differences between families can affect the relationship
  between overlap and phylogenetic distance at the network level. 
  Interestingly, in our analyses the plant families associated with
  the steepest relationships between niche overlap and phylogenetic
  distance at the network level did not show particularly steep
  relationships within themselves. This result suggests
  that it is not just which plant families are present but the additional 
  relationships between the families that affects conservation of
  interactions at the network level and is consistent with previous
  work showing that the shape of phylogenetic tress, as well as the
  phylogenetic distances between species, can affect the strength
  of phylogenetic signal~\citep{Chamberlain2014a}. Our results 
  emphasise the importance of considering conservation of interactions at multiple scales.
  We hope that these results will help to guide future work investigating
  the genetic and environmental drivers underpinning these relationships.


\section*{Acknowledgements}

  We would like to thank the authors of the published networks used in this study. 
  We are grateful for the use of the Edward Percival field station in Kaikoura, 
  New Zealand in May, 2014. This research was supported by an NSERC PGS-D 
  graduate scholarship (to ARC), a Marsden Fund Fast-Start grant (UOC-1101) and a 
  Rutherford Discovery Fellowship, both administered by the Royal Society of New 
  Zealand (to DBS), a BlueFern HPC PhD scholarship (to NJB), and by
  the Allan Wilson Centre (to GVDR).


\section*{Author Contribution}

  ARC, DBS, GVDR, and NJB designed the research. ARC, JAT, and CJW collected published data. [[Add MSC helpers!]]
  ARC and GVDR performed the analyses. All authors contributed to the manuscript.

\end{spacing}

\newpage
\bibliographystyle{newphy}
\renewcommand*{\bibfont}{\raggedright}
\bibliography{noisn}


\newpage
\section*{Tables}

  \begin{table}[!h]
  \caption{\small Change in log odds (per million years of phylogenetic distance) of a pair of plants in the same family sharing a herbivore.}
  \label{family_slopes_ph}
  \begin{tabular}{|l  rr|}
  \hline
    Family & Change in log odds & $P$-value \\
    \hline
    \emph{Asteraceae} & -1.73 &  0.550 \\
    \emph{Euphorbiaceae} & -19.2 & \textbf{\textless0.001} \\
    \emph{Fabaceae} & 18.7 &  \textbf{0.046} \\
    \emph{Melastomataceae} & -13.2 & \textbf{0.022} \\
    \emph{Moraceae} & -2.13 & 0.092 \\
    \emph{Nothofagaceae} & -595 & \textgreater{0.999} \\
    \emph{Pinaceae} &  -25.8 & 0.733 \\
    \emph{Poaceae} & -4.50 & \textbf{0.020} \\
    \emph{Rubiaceae} & -8.16 &  \textbf{0.006} \\
  \hline
  \end{tabular}
  \smallskip
  \footnotesize

  Nine plant families were sufficiently diverse in our  dataset to permit this analysis
  (see \emph{Materials \\and Methods} for details). For each pattern of overlap, we show the change
  in log odds per million years \\and the associated $P$-value. Statistically significant values are
  indicated in bold. \\

  \end{table}

  \begin{table}[!h]
  \caption{
  \small Change in log odds (per million years of phylogenetic distance) of a pair of plants in the same family sharing a pollinator.}
  \small
  \label{family_slopes_pp}
  \begin{tabular}{|l  rr|}
    \hline
    Family  & Change in log odds & $P$-value \\
    \hline
    \emph{Adoxaceae} &  -65.8 & 0.163 \\
    \emph{Amaryllidaceae} &  -17.9 & \textbf{0.015} \\
    \emph{Apiaceae} &  10.9  & \textbf{0.006} \\
    \emph{Apocynaceae} &  -6.96  & \textbf{0.037} \\
    \emph{Asparagaceae} &  -6.23  & 0.189 \\
    \emph{Asteraceae}* &  -1.47  & \textbf{\textless0.001} \\
    \emph{Berberidaceae} &  -1.48$\times10^3$ & \textgreater0.999 \\
    \emph{Boraginaceae} &  -5.15  & \textbf{\textless0.001} \\
    \emph{Brassicaceae} &  -11.2 & 0.072 \\
    \emph{Calceolariaceae} &  156 & 0.998 \\
    \emph{Campanulaceae} &  334 & 0.999 \\
    \emph{Caprifoliaceae} &  0.310 & 0.959 \\
    \emph{Caryophyllaceae} &  2.09 & 0.644 \\
    \emph{Cistaceae} &  -11.4 & \textbf{\textless0.001} \\
    \emph{Convolvulaceae} &  -1.84  & 0.837 \\
    \emph{Ericaceae} &  4.61 & 0.116 \\
    \emph{Fabaceae}* &  -12.9 & \textbf{\textless0.001} \\
    \emph{Geraniaceae} &  -3.31  & 0.624 \\
    \emph{Hydrangeaceae} &  0.057 & 0.982 \\
    \emph{Iridaceae} &  -27.9 & 0.078 \\
    \emph{Lamiaceae} &  -5.01  & \textbf{\textless0.001} \\
    \emph{Loasaceae} &  -865  & \textgreater0.999 \\
    \emph{Malpighiaceae} &  2.80 & 0.168 \\
    \emph{Malvaceae} &  -5.56  & 0.363 \\
    \emph{Melastomataceae}* &  5.19 & 0.577 \\
    \emph{Montiaceae} &  -1.12  & 0.870 \\
    \emph{Myrtaceae} &  8.55 & 0.071 \\
    \emph{Oleaceae} &  0.995 & 0.855 \\
    \emph{Onagraceae} &  -556  & \textgreater0.999 \\
    \emph{Orchidaceae} &  -14.5 & 0.145 \\
    \emph{Orobanchaceae} &  24.2  & 0.326 \\
    \emph{Papaveraceae} &  -11.2 & 0.511 \\
    \emph{Phyllanthaceae} &  9.99 & 0.433 \\
    \emph{Plantaginaceae} &  -8.48  & \textbf{0.001} \\
    \emph{Poaceae}* &  69.2  & \textbf{0.003} \\
    \emph{Polygonaceae} &  -14.8 & \textbf{\textless0.001} \\
    \emph{Primulaceae} &  14.9  & 0.343 \\
    \emph{Ranunculaceae} &  -38.0 & \textbf{\textless0.001} \\
    \emph{Rosaceae} &  0.759 & 0.735 \\
    \emph{Rubiaceae}* &  -13.0 & \textbf{0.026} \\
    \emph{Salicaceae} &  -1.90  & 0.545 \\
    \emph{Sapindaceae} &  821 & 0.999 \\
    \emph{Saxifragaceae} &  -0.092  & 0.992 \\
    \emph{Solanaceae} &  -21.9 & 0.189 \\
    \emph{Tropaeolaceae} &  192 & 0.997 \\
    \emph{Verbenaceae} &  -9.03  & 0.627 \\
    \emph{Violaceae} &  -0.487  & 0.974 \\
  \hline
  \end{tabular}
  \smallskip
  \footnotesize

    We were able to fit these models to 47 plant families (see \emph{Materials and Methods} for     details). 
    Families \\ marked with an asterisk were also sufficiently diverse in herbivory networks. 
    Statistically significant values \\ are indicated in bold. 

    \end{table}

\clearpage

\section*{Figures}

  % \begin{figure}[!h]
  %   \begin{center}
  %     \centerline{\includegraphics*[width=.75\textwidth]{Figures/Sketches/methods_breakdown.eps}}
  %   \end{center}
  %    \caption{Visual depiction of our decomposition of pairwise niche overlap of plants' interaction
  %    partners. \textbf{(a)} First, consider the representation of any pollination or herbivory network
  %    as an adjacency matrix. Here, filled cells indicate an interaction between a particular plant 
  %    (letters on rows) and an animal (numbers on columns). \textbf{(b)} For a given pair of plants (e.g., plants 
  %    A and B), we then considered the set of animals that interact with at least one of the focal
  %    plants. Taking each pair of animals in this set in turn, we assigned the resulting quartet (the
  %    two focal plants plus two animals) to one of three patterns of overlap.
  %    In the \textbf{total overlap} pattern, both plants interact with both animals. 
  %    In the \textbf{partial overlap} pattern, one plant interacts with both animals and the other plant
  %    interacts with only one.
  %    Finally, in the \textbf{no overlap} pattern each animal interacts with only one plant; note that 
  %    this includes cases where both animals interact with the same plant (e.g., animals 1 and 5 and plant A)
  %    as well as cases where each animal interacts with a different plant (e.g., animals 1 with plant A and animal
  %    4 with plant B). \textbf{(c-e)} The number of times each pattern occurred was used to summarise the pairwise niche 
  %    overlap between the two plants and then related to their phylogenetic distance.}
  %   \label{overlap_patterns}
  % \end{figure}


  \begin{figure}[!h]
    \begin{center}
      \centerline{\includegraphics*[width=.75\textwidth]{Figures/dataplots/scaled_regression_lines_full_color.eps}}
    \end{center}
     \caption{Results of a mixed-effects logistic regression of pairwise niche overlap
     against phylogenetic distance for plants in 11
     herbivory networks (top; green) and 59 pollination
     networks (bottom; purple). In both network types, the probability of a pair of plants
     sharing an interaction partner increased with increasing phylogenetic distance (thick,
     dark lines). There was substantial variation among networks (thin, pale lines) of both types. The slope of the regression for each network was significantly more extreme than that obtained from 999 permutatations of that network (slopes obtained from the permuted networks ranged between -1.34$\times10^{-12}$ and 9.19$\times10*{-13}$).}
    \label{within_network_regression} 
  \end{figure}

  \begin{figure}[!h]
    \begin{center}
      \centerline{\includegraphics*[width=.75\textwidth]{Figures/dataplots/observed_vs_random.eps}}
    \end{center}
     \caption{The slopes of the mixed-effect logistic regression of pairwise niche overlap against phylogenetic distance was signficantly different from 0 for each network. Here we show, for plant-herbivore (PH) and plant-pollinator (PP) networks, the observed slopes (depicted by green squares or purple diamonds, respectively). The maximum and minimum slopes obtained from 999 permutations of each network are depicted by thin, black lines. For both network types, the slopes obtained from permuted networks were always very close to 0.}
    \label{obs_vs_random} 
  \end{figure}

  \begin{figure}[!h]
    \begin{center}
      \centerline{\includegraphics*[height=.62\textheight]{Figures/dataplots/Family/allfams_full.eps}}
    \end{center}
     \caption{Change in log odds of observing pairwise niche overlap per million
     years of divergence time between a pair of plants
     in the 14 plant families where there was a significant
     relationship between log odds of sharing interaction
     partners and phylogenetic distance.
     Families in pollination networks are indicated
     by dark purple diamonds while families in herbivory
     networks are indicated by pale green circles.
     Note that changes in log odds are analogous to the 
     slopes of the regression lines from Eq. 2-3
     in logit-transformed space and represent
     the change in the probability of observing shared 
     interaction partners per million years of divergence time. We also
     show the slope of the relationship between the
     log-odds of observing each overlap pattern and 
     phylogenetic distance across all plant families
     in herbivory (pale, green horizontal line) and
     pollination (dark, purple horizontal line) networks.
     % See~\emph{Figure S1; Supporting Information} for more details.
     The phylogenetic tree below the plots indicates the
     relatedness between plant families. Error bars represent 95\% confidence
     intervals.}
    \label{within_family_regression}
  \end{figure}

\end{document}
